\documentclass[a4paper,12pt,%
headsepline,%
numbers=noenddot,%entfernt die Punkte nach den Gliederungsziffern
]{scrreprt}

\usepackage[T1]{fontenc}
\usepackage[utf8]{inputenc}
\usepackage[ngerman]{babel}
\usepackage{setspace} 



%-----------Format-------------
%\usepackage{titlesec} %Titel Format bearbeiten
\usepackage{setspace} 
\setstretch{1,15} % 1,5 Zeilenabstand, [singlespacing] oder [doublespacing] auch möglich

%-----------Grafiken und Tabellen-------------
\usepackage{float} %float-Objekte mit Großbuchstaben an Stelle behalten
\usepackage{caption} % Beschriftungen von Bildern & Tabellen bearbeiten
\captionsetup{format=plain,
	font=small,
	labelfont=bf}
\captionsetup[table]{singlelinecheck=false}

\usepackage{graphicx} % Grafiken einbinden

\usepackage{booktabs} % Tabellen ordentlich, toprule, midrule, cmidrule, bottomrule
\usepackage{tabularx} % Tabelle eine bestimmte Breite vorgeben, automatischen Zeilenumbruch innerhalb der Spalten
\usepackage{threeparttable} % Beschriftung von Tabellen auf Tabellenbreite, Fußnoten an Tabellen uvm; muss um die tabular Umgebung herum gesetzt werden

%-----------Listen-------------
\usepackage{outlines} %Verschachtelte Listen deutlich einfacher
\usepackage{enumitem} %Anpassbare Enumerates/Itemizes, Listen global konfigurieren
\setitemize{nosep} %Abstand zwischen Items, verwende auch itemsep=0p

%-----------weiterer Komfort-------------
\usepackage{textgreek} %Griechische Buchstaben mit \textalpha etc.

%----------überprüfen------------
\usepackage{microtype}
\usepackage[per=slash,decimalsymbol=comma,loctolang=ngerman]{siunitx}
\sisetup{detect-all} %SI-Einheiten in "normaler" Schriftart

%----------zuletzt laden---------
\usepackage[%pdflatex,%
pagebackref=true,%
pdfauthor={Ink Inc.},%
pdftitle={Wissenssammlung},%
pdfsubject={Handbuch}%
]{hyperref} %







%----------eigene Befehle--------
\newcommand{\npref}[1]{\nameref{#1} (S. \pageref{#1})} 
%wenn \npref{referenz} aufgerufen wird, wird automatisch eine Sequenz eingefügt: "`Name (s. X)"' zB "`Sylvan (S. 3)"'
%das sorgt dafür, dass wir nicht im ganzen Dokument die Namen ändern müssen, wenn wir etwas umbenennen
%willst du nur den Namen, ohne einen Hinweis auf die Seite (zB weil du das schon weiter oben im Text gemacht hast)






\begin{document}

\begin{titlepage}
	\centering
	\vfill
	\Large{\textbf{Stand {\today}}} \\ \bigskip
	\vfill
	\textsf{\textbf{\Huge{Wissenssammlung}}} \\ \bigskip
	\Large{\textbf{zur Übersicht der Geschichte von MMG der Ink Inc.}} \\ \bigskip
	\vfill
	\includegraphics[width=\linewidth]{Abbildungen/logo.png} \\ \bigskip
	\vfill
	\large
	bearbeitet von\\
	Medsmiley, Thabb, Mine Schokokeks, MiMi, LuckerStyle, Dommmmmi \\ \bigskip
	Concept-Art von Fossor, Kriluny
\end{titlepage}

\pagenumbering{Roman}
\tableofcontents
\listoffigures



%----------Inhalt-------------
\chapter*{Einleitung}
\pagenumbering{arabic}
Dieses Dokument dient zur Sammlung und Aufbewahrung allen beschlossenen Wissens bezüglich des MMG Spiels der Ink Inc. Die Lore soll einen guten Hintergrund geben, damit eine Vorstellung der Welt existiert. Damit lässt sich dann die Gesellschaft aufbauen, in der der Spieler seinen Charakter das Abenteuer erleben lässt. \\
Ebenso werden hier Mechaniken erläutert, die in das Spiel eingebaut werden sollen. \\

Nichts zu suchen haben hier detaillierte NSC- oder Questbeschreibungen. Dieses Netzwerk aus Beziehungen erfassen wir in einer eigens dafür erstellten Plattform. 

\part{Weltenbau}
\chapter{Karte \& Ökologie}
\href{http://www.weltenbau-wissen.de/sci-fi-erschaffen-fantasy-welt-erstellen-einstieg/}{Welt erstellen} \\
\href{http://www.weltenbau-wissen.de/2015/10/weltenbau-fragenkatalog-oekologie-biologie/}{Fragenkatalog} \\
\href{http://www.weltenbau-wissen.de/2015/01/weltenbau-mit-weltkarte-karte-zeichnen-tutorial/}{Weltkarte zeichnen} \\
\href{https://inkarnate.com/}{Weltenbautool}

\section{Sonnensystem}
\subsection{Planetensystem}
\begin{itemize}
	\item unsere Erde ist ein zwei-Planeten System - ähnlich wie wir es haben mit Erde und Mond. Entstehung auch ähnlich, allerdings ist unser "Mond" deutlich größer: beide Planeten haben fast die gleiche Masse und natürlich praktisch die gleichen Elemente
	\item die zwei Planeten sind kleiner als die Erde. Bzw. müssen sie halt mindestens so groß sein, dass sie eine Atmosphäre halten können: \\
	Es ist ein Problem, wenn unsere Planeten zu klein sind... Man geht davon aus, dass sich das Leben auf Planeten mit unterschiedlicher Größe auch deutlich anders entwickelt hätten (Gravitation ist hier ausschlaggebend). Tatsächlich bewegen sich alle als erd-ähnlich und potenziell bewohnbar eingestuften Planeten, die wir bisher entdeckt haben zischen 0,6 und 2,5 Erdmassen und einem Radius von 0,9 bis 1,4-fachen des Erdradius. Merkur hat 0,05 Erdmassen und nur einen Drittel des Erdradius, ist also viel viel zu klein um jemals Leben wie auf der Erde entwickeln zu können, selbst wenn er in der habitablen Zone läge. Der Mars hat übrigens 0,1 Erdmassen und den 0,5 fachen Durchmesser. Einzig die Venus läge mit 0,8 Erdmassen und etwa gleichen Radius in der vielleicht gerade noch so okayen Range. Wir werden also wohl oder übel mit fast-erdgleichen Planeten arbeiten müssen, zumindest wenn wir es korrekt halten wollen.
	\item dadurch, dass sie sich so dicht beieinander befinden, erfolgte nach der Entstehung des Lebens auf einem der Planeten die Übertragung auf den anderen durch Meteroiten-Einschläge, die ihrerseits Brocken in das All hinausschießen ließen, von denen halt welche mit Leben drauf auf den anderen Planeten auftrafen (da sie so dicht beieinander sind)
	\item im Laufe der Zeit ist das Gleiche erfolgt wie bei unserem Mond: die Drehung um die eigene Achse ist so, dass sich die Planeten immer mit der gleichen Seite angucken. Und die Drehung insgesamt im Doppel-Planeten-System ist jeweils auch so, dass es einen "normalen" Tag-Nacht-Rhythmus gibt
	\item "normal" heißt hier zB gleiche Stunden am Tag wie bei uns oder etwas mehr -- aaaber die Entfernung von der Sonne und die Bahn ist so, dass das Jahr deutlich länger ist. Wir haben also deutlich längere Jahreszeiten - was uns zugute kommt, wenn man bedenkt, dass wir keinen Winter und Herbst noch einfügen wollen :smile:
\end{itemize}

\subsubsection{Gara} \label{sec:planet-zwilling}

\subsubsection{Serro} \label{sec:mond}

\section{Andar} \label{sec:planet}
\section{Kontinent}
\begin{itemize}
	\item Ein Grenzreich ist ein kleiner bis mittlerer Staat, dem es wie Kambodscha unter den Roten iwas geht -> gerade mit sich selbst beschäftigt
\end{itemize}

\subsection{Gebirgszug}
\subsection{Gigantus Wald} \label{formation:gigantus}
\begin{itemize}
	\item wie bekannt, können auch Pflanzen die ihnen innewohnende Magie nutzen. Eine Art (!!) von Bäumen haben folgendes entwickelt: sie ermöglichen mittels der Magie die Wasserversorgung der oberen Baumabschnitte gegen die Gravitation. Das Größeneinschränkende Element normaler Bäume ist damit weg. Daher konnten die Bäume extrem groß werden, bevor weitere Prozesse ihr Wachstum stoppten und hatten damit den Platz an der Sonne sicher
	\item  diese Garganten (Working title) erreichen Höhen von 500m und ähnlich wie aus dem Dschungel bekannt, bilden sich somit verschiedene vertikale Lebensbereiche
	\item der Boden ist bedeckt mit allerlei Gehölz und Sträuchern, die wenig Sonnenlicht brauchen, und ganz vorne voran riesigen Pilzen, die in Symbiose mit den riesigen Bäumen ebenfalls ein gigantisches unterirdisches Netzwerk bildeten und nun gigantische Fruchtkörper (:mushroom:  das da) ausbilden können
	\item die ersten kräftigen Äste der Garganten haben stark grüne, mit viel Chlorophyll gefüllte Blätter, die jegliches Licht, das durch die oberen Schichten kommt, aufnehmen kann. Für unsere Verhältnisse stehen die Bäume zwar ewig weit auseinander, doch auf ein normales Größenverhältnis reduziert nicht. Ihre Äste können sich an den Spitzen erreichen und bilden so mittels der Äste und den riesigen Laubblättern eine Art zweiten Boden, beginnend in 200m Höhe
	\item durch die Ausmaße der Äste sammelt sich Erde hier und dort und es wachsen normale Bäume auf dieser Ebene, allerdings nicht zu weit vom Stamm der Garganten entfernt. Ebenso finden sich hier normale Büsche etc. Schlingpflanzen ziehen sich um die dicken Äste und es bildete sich so im Laufe der Zeit in dicker "Boden" - mit unerwartete Löchern hier und da
	\item Da wir uns hier schon in der Krone der Bäume befinden, ist ab dieser zweiten Schicht ein langsamer Verlauf in den folgenden 300m: Die Baumkrone begünstigt in ihren Ausmaßen das Wachsen anderer Bäume, Sträucher, Kräuter etc, insbesondere parasitischer Lebensformen wie Schlingpflanzen und Misteln. Auch einige "normale Bäume" wachsen nicht nur auf den Erdansammlungen, sondern auch in das Holz der Äste hinein. Insgesamt bildet sich so wie eine Art Gerüst, welches aus festen "Platformen", aus festen und wackligen Wegen horizontal, vertikal und schräg, aus Bereichen lockerer Vegetation (wie Lianen) und aus "Lichtungen" (3D, nicht 2D) besteht.
	\item nach oben hin werden die Blätter der Garganten immer lichtdurchlässiger. Die untersten Blätter sind mehrere Meter dick und dunkelgrün vor Chlorophyll. Die obersten sind nur ein paar Zentimeter dick, kleiner, leicht hellgrün gefärbt und ansonsten Lichtdurchlässig. Das sorgt dafür, dass das Licht bis unten hin durchkommt und effizient genutzt werden kann, um den riesigen Baum (und seine Parasiten) zu versorgen
	\item ebenso wie die Fauna, hat sich hier auch viel Getier angesammelt und angepasst in den verschiedenen Stufen. So nisten bestimmte Vögel nur in den oberen 50m der Baumkronen. Auch ein paar Menschen fanden sich hier vor langer Zeit ein und haben sich an das Leben in den Baumkronen (von 200-350m etwa) angepasst. Es ist die Heimat der \npref{rasse:sylvan}.
	\item allerdings ist dann doch der einschränkende faktor zum einen die stabilität von holz, bevor der baum unter der eigenlast zusammenbricht, und die biegsamkeit von holz, bevor höhenwinde den baum zerbrechen. gegen letzteres könnte der baum natürlich einfach sehr dick sein. mit werten, die ich jetzt auf die schnelle für holz gefunden hab, sollte das ganze bei wenigen hundert m höhe schon instabil werden. man könnte vielleicht argumentieren, dass der baum nicht nur wasser besser transportieren kann, sondern auch mineralstoffe, um seine stabilität zu erhöhen. dann hätte man gleich ein inhärent sehr hartes holz in der welt. bei dieser höhe ist die breite auch fast egal (solange sie wenige meter überschreitet)
\end{itemize}

\section{Land}
\section{Tal}
\subsection{Teich 1}
\subsection{Teich 2}
\subsection{Großer Wald}
\subsection{Klippe am Eingang}

 
\chapter{Magie}
\section{Das Konzept der Magie}
\subsection{Was ist Magie?}
In diesem Universum ist Magie eine besondere Form der Energie und ist ein Nebenprodukt bei Energieumwandlungen.
Es entsteht also überall dort Magie, wo irgendeine Form der Energie in eine andere umgewandelt wird - was sowohl in der belebten als auch der unbelebten Natur ständig der Fall ist.
So führen z.B. Steinschläge oder Lawinen zu einem plötzlichen Auftreten einer gewissen Menge von Magie, während ein Vulkanausbruch hingegen eine ziemliche Masse an Magie produziert. 
Ja schon allein die Plattentektonik sorgt für das Vorkommen von Magie.
Da Magie eine Form der Energie ist, kann sie deshalb auch selbst wieder zurück in andere Energieformen umgewandelt werden.
Im Allgemeinen "`diffundiert"' die erzeugte Magie nämlich von ihrem Ort der Entstehung hinweg in die Umgebung, da es nichts gibt, was sie halten würde.
Dort wird sie nach und nach wieder umgewandelt.


\subsection{Anpassung der belebten Natur}
Es ist also kein Wunder, dass Magie hier auf \nameref{sec:planet} nichts besonderes ist, und dass sich das Leben dementsprechend auch entwickelte - denn insbesondere in und um Zellen erfolgt am laufenden Band die Umwandlung von Energie zwischen verschiedenen Formen.
Daher haben Zellen eine Möglichkeit entwickelt, Energie in Form von Magie zu halten und zu einem gewissen Grad anzusammeln.
Bei Eukaryoten (Pflanzen, Pilze, Tiere) ist dies eine Art Vakuole. 
Die maximal zu haltende Menge ist artabhängig und angeboren.
Ist diese Menge überschritten, so diffundiert alles Überschüssige wieder aus der Zelle in die Umwelt.
Die so gehaltene Menge an Magie wird auch als Mana oder Manapool bezeichnet.

Eine Verbildlichung lässt sich mit einer Zisterne darstellen: diese füllt sich mit dem Wasser vom Dach des Hauses, bis sie voll ist.
Danach läuft sie über und das Wasser verteilt sich in der Gegend.

Ursprünglich war dies ein Vorteil bezüglich des Überlebens der Zelle: im größten Notfall, wenn die Zelle keine Energiezufuhr jegweder Art erhält, dann kann sie ein energieintensives Notprogramm initialisieren, welches ihr die Magie als tatsächliche Energiereserve zu nutzen ermöglicht.
So kann die Zelle einen kurzen Zeitraum ohne Versorgung überbrücken.
Im Laufe der Evolution haben viele Lebewesen dann Mechanismen entwickelt, um die ihnen innewohnende Magie in ihrem Sinne auf einem intuitiven Level umzuwandeln bzw einzusetzen.
Bei Tieren kann man sich das wie eine Art zweites Nervensystem vorstellen, welches sich an den Adern entlang durch den ganzen Körper zieht und vom Rückenmark aus kontrolliert wird.
Dieses Geflecht wird aus besonderen Zellen gebildet, die selbst keine Magie speichern, sondern die Magie aus den restlichen Zellen des Körpers abziehen.
Diese gelangt somit in das Netzwerk, wo sie über das Rückenmark gezielt kontrolliert werden kann.

In welcher Weise sie die Magie kontrollieren können, ist genetisch festgelegt und hat sich evolutionär entwickelt.
Der Einsatz erfolgt im intuitiven Sinne wie ein Reflex oder eine Reaktion auf etwas, das passiert, oder unterstützt eine Handlung. 
So gibt es Pflanzen, die zur Abwehr von Fressfeinden kleine Elektroschocks verteilen, zum Anlocken von Bestäubern mit Licht spielen, oder Prädatoren, die mittels Infrarot-Magie ihre Beute ohne Probleme von der Umgebung unterscheiden können.

Jeder Mensch beherrscht die Magie auf diesem Level und zeigt sich erstmals im Bereich von 4-7 Jahren.
Im Verlauf der Differenzierungen der verschiedenen Menschenarten haben sich unterschiedliche Ausprägungen durchgesetzt oder sind erst entstanden.


\subsection{Aktive Verwendung}
Einige Lebewesen haben zudem die Fähigkeit entwickelt, die Magie aktiv in ihrem Interesse einzusetzen.
Bei Tieren bedeutet das, dass sich auch das Gehirn bei der Steuerung einschaltet.
Dabei wird sie auf ein Ziel ausgerichtet und dort die bestehenden Zustände geändert.
Es muss allerdings eine deutlich größere Menge an Magie eingesetzt werden, als der Vorgang bzw. die Änderung eigentlich an Energie benötigen würde.
So hat auch der Mensch schließlich festgestellt, dass er mithilfe seines Willens und Konzentration dazu in der Lage ist, diese Möglichkeiten auszubauen und zu verstärken.
Nur Personen, die sich intensiv mit ihren Fähigkeiten auseinandersetzen, viel meditieren, experimentieren und Verständnis suchen, nur diese Personen lernen, das Tauschverhältnis zu reduzieren und immer mehr zu einem halbwegs äquivalenten Austausch zu kommen.
Das führt dazu, dass sie mit dem ihnen angeborenen Manapool mehr und stärkere Dinge bewirken können.
Ihre einzigen Grenzen sind dabei durch ihre Gene, ihre Vorstellungskraft, ihre Konzentrationsfähigkeit und den magischen Widerstand anderer Lebewesen gesetzt. 
\\ \\
Es gibt ein paar Prokaryoten, die sich die Magie als tatsächliche dauerhafte Energiequelle erschlossen haben und sie ständig direkt zur Herstellung von ATP (Adenosintriphosphat = "`Energie der Zelle"') nutzen können.
Bisher ist noch kein Fall von Symbiose oder gar Endosymbiose ähnlich wie mit den Mitos (\textalpha-Proteobakterien $\rightarrow$ Mitochondrien) oder Chloros (Cyanobakterien $\rightarrow$ Chloroplasten) bekannt, aus dem höher entwickelte Lebewesen herausgekommen sind, da dies vor nicht allzu langer Zeit (in Zeiträumen der Evolution) entstand.


\subsection{Natürlicher magischer Widerstand}
Ist es im Allgemeinen nicht schwer, den eigenen Körper und die Umgebung zu beeinflussen, so ist das Verändern von Zuständen in anderen Körpern ein ganz anderes Thema.
Höher entwickelte Lebewesen, die ein Verständnis für den eigenen Körper oder sogar ein Ich-Bewusstsein erstanden haben, besitzen dadurch einen Art natürlichen magischen Widerstand gegen Änderungen, die in ihrem Körper erfolgen sollen. 
Dieser ist umso stärker, je ausgereifter das Bewusstsein ist.
Das führt dazu, dass nur wirklich mächtige und studierte Individuen es schaffen, diese Barriere zu überwinden und Magie innerhalb solcher Körper zu wirken.



\section{Magie-Kontrolle}
\subsection{Hintergrund}
Magie kann eingesetzt werden, um bestimmte physikalische oder chemische Prozesse zu verändern - in einem gewissen Rahmen, der genetisch vererbt wurde.
Somit bedeutet die Beherrschung einer Magie-Art eigentlich, dass das Wesen dazu fähig ist, die Magie kontrolliert in diese bestimmte Art Energie umzusetzen.

\subsection{Stufen der Kontrolle}
Bei den folgenden verschiedenen Arten der Magie-Kontrolle, auch Magiearten genannt, ist dargestellt, wie die intuitive Nutzung dieser Eigenschaft aussieht oder wie Begabte mit ihr umgehen können.
Meisterliche Magier erreichen noch ganz andere Level. 
In der Regel sind sie schon in hohem Alter und haben sich ihr Leben lang mit der Verbesserung, dem Lernen und Ausprobieren beschäftigt, weshalb es nur wenige von ihnen gibt, die so weit kommen. 
Neben der Intensivierung und Verbesserung vorheriger Zauber gelingt ihnen auch eine Manipulation ganz anderer Art: 
Sie können die natürliche Resistenz des Körpers (siehe oben) umgehen.



\section{Magie-Arten}
Die Informationen über die Verbreitung der Magiearten unter den Lebewesen und ihrer Bekanntheit in \nameref{sec:land} finden sich einerseits bei jeder Art und andererseits in Tab. \ref{tab:magie-verbreitung} zur Übersicht. 

Die Anordnung der Magiearten ist alphabetisch, um einen schnellen Zugriff zu ermöglichen.

\begin{table}[htbp]
	\centering
	\caption{Verbreitung der Magiearten unter den Lebewesen allgemein und unter den Menschenarten.}
	\label{tab:magie-verbreitung}
	\begin{threeparttable}[\linewidth]
		\begin{tabularx}{\textwidth}{l|ccX}
			\toprule
			\textbf{Art} & \textbf{Hominini} & \textbf{Mensch} & \textbf{Bemerkungen} \\
		    \midrule
			Absorption & - & - & Bisher nur bei Mikroorganismen \\
			Druck & X & X & Zweithäufigste unter den Menschen, auch im Tierreich stark verbreitet \\
			Elektrizität & X & X & Mäßige Verbreitung unter den Menschen, vor allem schlecht ausgenutzt, da Elektrizität ihnen noch kein richtiger Begriff ist. Vor allem bei Pflanzen und Tieren \\
			Licht & X & X & Mäßige Verbreitung unter den Menschen \\
			Proliferation & X & X & Unter den Menschen sehr selten \\
			Strahlung & - & - &  \\
			Temperatur & X & X & Am weitesten verbreitet \\
			Verhärtung & X & - & Sylvan \\
			Vibration & X & - & Zwerge und Undar \\
			\bottomrule
		\end{tabularx}
	\end{threeparttable}
\end{table}



\subsection{Absorption oder Nullmagie}\label{sec:nullmagie}
Diese Art der Kontrolle sorgt für einen ständigen Entzug von Magie aus der Umwelt, da zumindest beim Menschen die eigenen Zellen nicht mehr fähig sind, diese zu halten und somit versuchen, auszugleichen. 

\subsubsection{Verbreitung}
Zu Beginn des Spiels beherrschen nur Mikroorganismen diese Fähigkeit. Am Ende wurde eine neue Art Mensch geschaffen, die diese Fähigkeit ebenso beherrschen. Sie ist daher anfangs vollkommen unbekannt.

\subsubsection{Bedeutung im Spiel}
Zum aktuellen Zeitpunkt nicht existent, denn sie entsteht erst bei der Gottwerdung in den Überlebenden.

\subsubsection{Intuitive Nutzung}
Die Bakterien nutzen es als Energiequelle mittels eines speziellen Zellorgans. Es lässt sich ähnlich vorstellen wie die Nutzung von Sonnenenergie mittels Chloroplasten oder chemischer Energie mittels Mitochondrien. 

Die Menschen werden durch das heftige Ziehen sämtlicher Magie relativ unbeeinflussbar durch Magie, die direkt auf sie gewirkt wird oder auf das, was sie berühren. Sie sind jedoch unfähig, selbst Magie zu wirken.

\subsubsection{Erlernbare Steigerungen}
Es wird sehr lange dauern, bis die Menschen begreifen, dass auch dies eine Möglichkeit zur Steigerung bietet.
\begin{itemize}
	\item Aktives Absorbieren von Magie aus Lebewesen, was diese daran hindert, ihre Magie einsetzen zu können.
	\item ...
\end{itemize}

\subsubsection{Meisterliche Beherrschung} 
\begin{itemize}
	\item ...
\end{itemize}




\subsection{Druck}\label{sec:druckmagie}
Diese Eigenschaft erlaubt die Kontrolle über den lokalen Druck und damit verbunden die Entstehung von Wind. Es existiert die präzisere Unterform \nameref{sec:vibrationsmagie}, sowie die weiterentwickelte Form \nameref{sec:haertungsmagie}.

\subsubsection{Verbreitung}
Ähnlich wie bei der thermischen Energie ist diese Form recht weit verbreitet. Die homininen Arten beherrschen diese Kontrolle. 

In unserem Land ist die Magieform allerdings nicht ganz so häufig wie die der Temperatur anzutreffen. Ansonsten ist sie gut bekannt.

\subsubsection{Bedeutung im Spiel}
Die \npref{sec:mc_diplomatin} hat diese Eigenschaft ab ihrem Schlüsselmoment. 

\subsubsection{Intuitive Nutzung}
Die Lebewesen können somit in einer Art Reaktion Gefahren, die zu nah an sie herankommen, von sich wegschubsen, ohne sie zu berühren, oder sich selbst aus einem gefährlichen Bereich heraus drängen. Ebenso können sie damit direkt um sich herum Wind erzeugen oder diesen negieren und sind daher von stürmischen Auswirkungen meist weniger betroffen.

Bei den Menschen findet sich eine zusätzliche Verwendung meist nur darin, lustige Effekte mit den Haaren hervor zu rufen.

\subsubsection{Erlernbare Steigerungen}
\begin{itemize}
	\item Es sind deutlich weitere und höhere Sprünge möglich.
	\item Das Erzeugen kleiner Windhosen bzw. Staubteufel.
	\item Es wird Druck auf andere Dinge angewendet, womit sie entweder auseinander gerissen oder zerquetscht werden.
	\item Gegenstände oder Lebewesen werden kontrolliert an Ort und Stelle gehalten oder durch die Gegend gedrückt.
	\item Der eigene Körper kann etwas über den Boden schweben.
	\item Einkommender Druck kann abgefangen und auf eine größere Fläche verteilt werden: zuerst nur bei größerflächigen Dingen wie Steinen oder Stöcke. Später auch gegen Waffen, die genau so was ausnutzen: Klingenwaffen. Dadurch wird weniger Schaden genommen.
	\item ...
\end{itemize}

\subsubsection{Meisterliche Beherrschung} 
\begin{itemize}
	\item Den Druck im Kopf erhöhen, bis er platzt.
	\item Den Blutdruck plötzlich extrem absenken, so dass der Gegenüber bewusstlos zusammenbricht. 
	\item Den Druck in der Umgebung so stark beeinflussen, dass sich dadurch das Wetter ändern kann. Damit lassen sich zB. Regenwolken lenken. 
	\item Eine Art Fliegen, aber mehr im Sinne eines erweiterten Schwebens.
	\item ...
\end{itemize}



\subsection{Elektrizität und Magnetismus}\label{sec:elektromagie}
Mit dieser Eigenschaft lassen sich elektrische Ströme und Ladungen beeinflussen. Ebenso kann dies für Elektromagnetismus genutzt werden.

\subsubsection{Verbreitung}
Dies ist weit unter Tieren und insbesondere Pflanzen verbreitet. Doch auch die Hominini besitzen diese Fähigkeit, allerdings ist ihnen Elektrizität oder im weiteren Sinne noch kein Begriff, weshalb das Ausbauen und Ausnutzen dieser Kontrolle recht schlecht ist.

\subsubsection{Bedeutung im Spiel}
Die \npref{sec:mc_diplomatin} hat diese Eigenschaft von Beginn an. 

\subsubsection{Intuitive Nutzung}
Die meisten Lebewesen nutzen Schockstöße zur Abwehr von Räubern und Fraßfeinden. Werden sie von Blitzen getroffen, so sterben sie nicht, sondern dienen als eine Art Ableiter. 

\subsubsection{Erlernbare Steigerungen}
\begin{itemize}
	\item Metallische (magnetische) Gegenstände von entsprechender Größe sind frei bewegbar.
	\item Metalle werden (ent-)magnetisiert.
	\item Das Erzeugen von elektrischer Energie, die an die Umwelt abgegeben werden kann.
	\item ...
\end{itemize}

\subsubsection{Meisterliche Beherrschung} 
\begin{itemize}
	\item Die Signale von Nervenzellen werden an ihnen selbst mittels Elektrizität weitergeleitet. Dieser Fluss lässt sich stören, womit zum Beispiel eine Reaktion des Gegenübers verhindert wird. Das ist allerdings recht unpräzise, da die Funktionsweise des Körpers noch nicht so weit bekannt ist. Durch das spüren der Ströme allerdings konnten einige Erkenntnisse gewonnen werden. So lässt sich zB sämtlicher elektrischer Fluss in einzelne Gliedmaßen ausschalten, wodurch diese nicht mehr bewegt werden können.
	\item Ebenso kann man damit das Gehirn von jemanden lahm legen und ihn damit effektiv töten.
	\item Das Herz wird von einem Nervenknoten, dem Sinus, kontrolliert. Bei Herzstillstand kann man es wieder zum schlagen bringen, indem man diesem Knoten einen elektrischen Impuls gibt, damit das Herz wieder zu arbeiten beginnt (siehe auch Defibrillator).
	\item ...
\end{itemize}



\subsection{Strahlung}\label{sec:strahlungsmagie}
Diese Eigenschaft ermöglicht die Kontrolle von Strahlung: Aussenden und Absorbieren. Das umfasst alle Strahlungsfrequenzen von  über ... bis hin zu y.
ermöglicht das Aussenden und Absorbieren von hochenergetischer Strahlung: energiereicher als Licht aka UV, alpha-, beta-, gamma-Strahlung. Aufteilung in verschiedene Bereiche, wie schon Licht abgetrennt ist? 

\subsubsection{Verbreitung}
...\\
Keine Verbreitung unter den Menschen.

\subsubsection{Bedeutung im Spiel}
Derzeit keine Bedeutung im Spiel.

\subsubsection{Intuitive Nutzung}
Lebewesen mit nackter Oberfläche bekommen keinen Sonnenbrand.

\subsubsection{Erlernbare Steigerungen}
\begin{itemize}
	\item Das Aussenden entsprechender Strahlung führt zu gehäuften Mutationen, was zu Tumoren führen kann.
	\item Erweitertes Sehen mittels UV-Strahlung?
	\item ...
\end{itemize}

\subsubsection{Meisterliche Beherrschung} 
\begin{itemize}
	\item Das Hervorrufen der Strahlungskrankheit.
	\item ...
\end{itemize}

\subsection{Licht und Dunkelheit} \label{sec:lichtmagie}
Diese Eigenschaft ermöglicht die Kontrolle von Licht bzw. dessen Abwesenheit.

\subsubsection{Verbreitung}
Eine recht weit verbreitete Kontrollfähigkeit unter Lebewesen. Bei den Hominini ist sie nicht so stark verbreitet wie andere.

\subsubsection{Bedeutung im Spiel}
Die \npref{sec:mc_spionin} beherrscht diese Eigenschaft. 


\subsubsection{Intuitive Nutzung}
Viele Lebewesen nutzen dies zur Attraktion von anderen Lebewesen, sei es um diese zu Nutzen oder als Beute. Einerseits Licht insbesondere bei Nacht, andererseits Illusionen um zu täuschen. \\
Man munkelt auch von einer Tierart, die sich so gut an die Umwelt anpassen, dass sie von ihrer Beute nicht entdeckt wird, bis es zu spät ist. 

Menschen nutzen dies insbesonders um sich selbst attraktiver zu machen.

\subsubsection{Erlernbare Steigerungen}
\begin{itemize}
	\item Das Erzeugen von Licht.
	\item Mit hellem Licht lassen sich Gegner blenden oder gar erblinden.
	\item Illusionen dienen zur Unterhaltung oder Ablenkung.
	\item Mittels Illusionen kann man sich dem Hintergrund anpassen, wodurch man deutlich schlechter zu sehen ist.
	\item ...
\end{itemize}

\subsubsection{Meisterliche Beherrschung} 
\begin{itemize}
	\item Wahre Meister des Fachs können das Licht so geschickt manipulieren, dass sie einem unaufmerksamen Auge gar als unsichtbar scheinen können.
	\item Die Kontrolle über die An- oder Abwesenheit von Licht in einem gewissen Bereich.
	\item ...
\end{itemize}



\subsection{Proliferation}\label{sec:proliferationsmagie}
Diese Eigenschaft ist eher chemischer Natur und erlaubt die Beeinflussung der Reaktionsgeschwindigkeit.

\subsubsection{Verbreitung}
Wenige Pflanzen- und Tierarten sind dieser Fähigkeit mächtig. Auch bei den Hominini ist es eine seltene Gabe. Aufgrund der Nachteile ist gar der Glaube verbreitet, dass diese Menschen ihr eigenes Leben geben, um andere zu heilen, weshalb nur höherrangige Leute davon profitieren dürfen.

\subsubsection{Bedeutung im Spiel}
Nur sehr wenig Individuen im Land haben diese Kontrollform in größerem Ausmaße geerbt. Diese sind hochgeschätzte Mitglieder des Ordens und erfreuen sich besten Umständen. Der Spieler erhält daher erst später Zugriff darauf, nachdem er entsprechend im System aufgestiegen ist.

\subsubsection{Intuitive Nutzung}
Diese Menschen leben kürzer als der Rest, da in ihrem gesamten Körper die Reaktionen sehr schnell ablaufen. Dadurch haben sie eine gesteigerte Reaktionszeit und heilen deutlich schneller.

\subsubsection{Erlernbare Steigerungen}
\begin{itemize}
	\item Erweitern der gesteigerten Reaktionen auf die nahe Umwelt: Pflanzen wachsen schneller.
	\item Zudem kann dies zur Heilung anderer eingesetzt werden.
	\item Allerdings können damit auch krankhafte Tumore hervorgerufen werden... 
	\item ...
\end{itemize}

\subsubsection{Meisterliche Beherrschung} 
Nur extrem wenige Menschen haben lange genug gelebt, um diese Stufe zu erreichen.
\begin{itemize}
	\item Das plötzliche starke Altern des Gegenübers.
	\item Das Verlangsamen der eigenen Alterung durch das Umkehren der obigen Prozesse. Das kehrt auch die anderen Folgen um. Es heißt, es liegt ein solcher Magier irgendwo in einer Art Winterschlaf und er wird aufwachen, wenn die Welt ihn wieder benötigt (Ref: König Barbarossa? Ggf. unter Märchen und Quests ausfeilen). %todo
	\item ...
\end{itemize}



\subsection{reine Energie}\label{sec:energiemagie}
...

\subsubsection{Verbreitung}
...\\
Keine Verbreitung unter den Hominini.

\subsubsection{Bedeutung im Spiel}
Derzeit keine Bedeutung im Spiel.

\subsubsection{Intuitive Nutzung}
...

\subsubsection{Erlernbare Steigerungen}
\begin{itemize}
	\item ...
\end{itemize}

\subsubsection{Meisterliche Beherrschung} 
\begin{itemize}
	\item ...
\end{itemize}



\subsection{Temperatur}\label{sec:temperaturmagie}
Diese Eigenschaft erlaubt die Kontrolle über die thermische Energie der Umgebung, sowohl das Erhitzen als auch das Abkühlen.

\subsubsection{Verbreitung}
Diese Manipulation der thermischen Energie ist uralt und daher weit verbreitet unter allen Arten von Lebewesen; Somit auch unter den homininen Arten. Sie ist gut bekannt.

\subsubsection{Bedeutung im Spiel}
Diese Magiekontrolle beherrscht der \npref{sec:mc_soldat}. Es ist ansonsten eine der am weitesten verbreiteten Arten unter der Bevölkerung. Allerdings wird sie unter Hominini nicht vererbt sondern ergibt sich aus der Kombination der Gene für Druck- und Luft-Kontrolle.

\subsubsection{Intuitive Nutzung}
Diese Lebewesen haben weniger Temperaturprobleme, da sie um sich herum eine kleine Blase erwärmter oder abgekühlter Luft schaffen können. Einige nutzen dieses Konzept auch zur Jagd, sowohl an Land als auch im Wasser, wo sie die Luft oder das Wasser stark erhitzen oder extrem abkühlen und damit für Verbrennungen oder Erfrierungen sorgen.

\subsubsection{Erlernbare Steigerungen}
Zuerst Beispiele für die Erhitzung:
\begin{itemize}
	\item Das Erhitzen von Gegenständen abseits vom eigenen Körper, insbesondere sehr gut Wärme-leitenden Materialien wie Metall. Dies wird zB. im Kampf ausgenutzt, um die Waffe oder die Rüstung des Gegners zu heiß zum Halten/Tragen werden zu lassen.
	\item Das Entzünden entflammbarer Objekte.
	\item Das Erschaffen kleiner Massen glühender Hitze, die weggeworfen werden kann. Beim Aufprall entzünden sich die vor Ort befindlichen Materialien. 
	\item Wandförmige Bereichen von Luft werden so stark erhitzt, dass alles, was sie berührt, schmilzt oder brennt.
	\item ...
\end{itemize}

Nun Beispiele für das Abkühlen:
\begin{itemize}
	\item Das Metall am Körper des Gegenübers erkalten lassen. Das kann zu Verbrennungen führen.
	\item Feuchte Körperteile lassen sich an gefrorenen Gegenständen, insbesondere metallischen, anfrieren.
	\item Flüssigkeiten auf dem Boden (idR Wasser) lassen sich gefrieren, womit Glatteis erzeugt wird.
	\item Feuer lassen sich durch einen gegensätzlichen Effekt löschen.
	\item ...
\end{itemize}

\subsubsection{Meisterliche Beherrschung} 
\begin{itemize}
	\item Das Blut/der Körper des Gegners kann zum Kochen gebracht werden.
	\item Körperteile des Gegners können erfroren werden.
	\item ...
\end{itemize}



\subsection{Verhärtung}\label{sec:haertungsmagie}
Mittels dieser Fähigkeit lassen sich Atombindungen stärken oder schwächen, wodurch ein Verhärtungs- oder Erweichungseffekt hervorgerufen wird. Als Unterform der \nameref{sec:druckmagie} wird dabei gleichzeitig ein starker negativer (Verhärtung) oder positiver (Erweichung) Druck von außen appliziert, wodurch sich das veränderte Material nicht plötzlich um den gleichen Faktor verkleinert oder vergrößert.

\subsubsection{Verbreitung}
Dies ist vor allem unter Tieren verbreitet und bei den Hominini können nur Sylvan auf diese Art Magie kontrollieren.

\subsubsection{Bedeutung im Spiel}
Derzeit keine Bedeutung im Spiel. Eventuell durch Tiere oder sollte man die Sylvan tatsächlich treffen können.

\subsubsection{Intuitive Nutzung}
Intuitiv nutzen Lebewesen dies, um den eigenen Körper zu stärken, wenn es um eingehende Verletzungen geht, zB. beim Kampf oder beim Fallen (Schockabsorption). Einige machen ihren Körper dadurch zeitweise etwas beweglicher, um durch kleinere Spalte hindurch zu passen. Dies ist auch anwendbar auf Oberflächen, die direkt berührt werden. Auf diese Art sorgen zB. die Sylvan dafür, dass sie auch bei etwas dünneren Blättern oder Ästen nicht durch krachen.

\subsubsection{Erlernbare Steigerungen}
\begin{itemize}
	\item Zerreißen von Dingen.
	\item Extremes Erhärten von Luft, um sie als stabile Wand zu nutzen.
	\item Bewusstes Weglassen des Drucks von außen führt zu einer Volumenänderung des betroffenen Bereichs. Dies bedeutet einerseits starkes Anwachsen oder extremes Verkleinern. Das Gewicht ändert sich dabei nicht. Die Weiterleitung von Signalen an Nervenzellen u.ä. ist allerdings auch beeinträchtigt: wenn größer, dann langsamer, wenn kleiner, dann schneller.
	\item ...
\end{itemize}

\subsubsection{Meisterliche Beherrschung} 
\begin{itemize}
	\item Erschaffen temporärer Löcher im Körper, durch die zB etwas durch kann, bevor dies wieder rückgängig gemacht wird.
	\item Anwendung in Alchemie: Stoffe werden dazu gebracht Verbindungen einzugehen, die normalerweise eine zu hohe Energiebarriere haben.
	\item ...
\end{itemize}



\subsection{Vibration bzw. Schall}\label{sec:vibrationsmagie}
Diese Fähigkeit ist eine Weiterentwicklung der \nameref{sec:druckmagie} im Sinne einer präziseren Manipulation.\\
Der Körper ist dazu in der Lage, verschieden-frequentige Vibrationen zu erzeugen. Häufig ging daher auch eine Evolution von feineren Sinnesorganen zur Rezeption dieser damit einher.

\subsubsection{Verbreitung}
Manche Lebewesen kontrollieren Magie auf diese Art, darunter auch die \nameref{rasse:zwerg} und \nameref{rasse:unda} der homininen Arten. Die Menschen beherrschen diese Verwendung der Magie nicht und da eine Reproduktion mit den anderen Arten nach der ersten Generation in einer Sackgasse steckt, bleibt ihnen nur die Erforschung.

\subsubsection{Bedeutung im Spiel}
Bedeutung findet sich hier keine weitere für die Menschen des Landes, nur in der Forschung. Bei den paar Zwergen, die unter den Menschen leben, sieht das natürlich anders aus.

\subsubsection{Intuitive Nutzung}
Diese Lebewesen nehmen Geräusche besser und in einer größeren Bandbreite wahr, bis hin zu einer Art Echolot-Technik. Allerdings können sie ihre Wahrnehmung dabei willentlich verschärfen oder abstumpfen. Ebenso können sie schon feinere Schwingungen über die Druckrezeptoren der Haut und Haare spüren und all diese Schwingungen auch aussenden.

Hominine nutzen das zur Kommunikation oder Orientierung in der Dunkelheit. 

\subsubsection{Erlernbare Steigerungen}
\begin{itemize}
	\item Die Wahrnehmung der Umgebung verstärkt sich extrem, teilweise werden keine Augen mehr benötigt um Dinge um sich herum zu lokalisieren. 
	\item Die Vibration wird zu einer Art kontrollierten Sprengung oder Zerstörung von (harten) Materialien genutzt.
	\item ...
\end{itemize}

\subsubsection{Meisterliche Beherrschung} 
\begin{itemize}
	\item Von gut bekannten Dingen oder Lebewesen kennen Meister des Faches die ihnen eigene Schwingungen bzw. Reaktion auf solche. Daher können sie solche auch über weite Entfernung aufspüren.
	\item Das Erzeugen von so lauten Geräuschen, dass sie kilometerweit zu hören sind. Das kann in nächster Nähe natürlich zu Gehörschäden führen.
	\item Gezielte Übertragung von Geräuschen über weite Entfernung, z.B. zur Kommunikation mit dem Rudel. \\ 
	Variante A: Erzeugung vor Ort und gerichtete (punktgenaue) Übertragung (auf einer Linie) zum Zielort. \\
	Variante B: Erzeugung beim Zielort.
	\item Knochen des Gegenübers zum Zerbröseln bringen.
	\item ...
\end{itemize}




\section{Die Genetik der Magie}
\begin{itemize}
	\item Magie wird vererbt und zwar (der Einfachheit halber) mendelsch. D.h. es existeren in jedem Tier (bei Pflanzen ist das alles komplizierter und bei Bakterien ganz anders) je ein Allel von der Mama und ein Allel vom Papa durch welches festgelegt ist, welche Magie man ausprägen kann. Wenn also beide Eltern homozygot (aka beide Allele gleich) Elementaristen sind, dann kann das Kind auch nur ein Elementarist sein.
	\item kurz gesagt: seltene Magie Arten sind rezessiv, häufige dominant. Ansonsten gibt es Magie-Arten, die sich so aufgrund der Kombination zweier dominanter Allele ausprägen
\end{itemize}
\subsection{Allele der Menschen unseres Landes} %todo hübsche Tabelle + Beschriftung!
Unter den Menschen in unserem Fokus sind 5 Gene verbreitet: \\ \\
\begin{tabular}{|c|c|c|c|c|}
	\hline
	& dominant-rezessiv & intermediär & kodominant* & Häufigkeit  \\ \hline
	Temperatur   &         -         &     DL      &      -      & sehr häufig \\ \hline
	Druck     &      DD, Dp       &      -      &     DE      &   häufig    \\ \hline
	Licht     &      LL, Lp       &      -      &     LE      &   häufig    \\ \hline
	Elektrizität  &      EE, Ep       &      -      &   LE, DE    & mittelmäßig \\ \hline
	Proliferation &        pp         &      -      &      -      &   selten    \\ \hline
\end{tabular} \\ *äußert, äußerst selten \\
\\
\textbf{Dominant-Rezessiv}: Alle Nachkommen in F1 haben den gleichen Phänotyp eines Elternteils. In F2 dann nur 75 zu 25 \% (zB. rote + weiße -> roten Blüten bei Erbsen). \\
\textbf{Intermediär}: Alle Nachkommen in F1 haben eine Mischform der Eltern, in F2 dann nur 50 \%. Je 25 \% sind reinerbig (z.B. rote + weiße -> rosa Blüten bei der Wunderblume). \\
\textbf{Kodominant}: Ähnlich wie intermediär, nur dass die Merkmale der Eltern separat und parallel ausgebildet werden (z.B. Blutgruppe AB).\\ \\
Davon nutzt der Soldat Temperatur, die Spionin Licht und die Diplomatin zuerst Druck und später auch Elektrizität.


\href{http://www.weltenbau-wissen.de/2015/12/magie-weltenbau-magiesystem-mystik-wissenschaft-teil-1/}{Magie soll Mystik behalten}\\
\href{http://www.weltenbau-wissen.de/2016/01/6-konsequenzen-magie/}{Konsequenzen von Magie}\\
\href{https://meisterperson.wordpress.com/2016/05/05/magie-fortschritt/?pk_campaign=pifeed\&pk_kwd=magie-fortschritt}{Magie und Fortschritt}
\chapter{Essenz}
\section{Was ist Essenz?}
\begin{outline}
	\1 Lebewesen besitzen eine Essenz, eine Art Seele (wird im Folgenden als Seele bezeichnet, um die Erklärung einfacher zu halten). 
	Diese Seele umfasst das, was dieses Lebewesen "sein eigen" nennt im Sinne des Körpers, der Existenz.
	Das ist im Normalfall der Körper. Könnte aber auch eine Prothese am Körper umfassen, wenn das Lebewesen diese als Teil seines Körpers begreift.
\end{outline}
\section{Lebensformen}
\subsection{Humanoide}
\subsubsection{Menschen}
\subsubsection{Zwerge}
\subsubsection{Halblinge}
\subsubsection{Sylvan}


\subsection{Tiere}
\subsection{Pflanzen}
\subsection{Mikroorganismen}


\part{Geschichte}
\chapter{Welt}
\chapter{Kontinent}
\section{Nachbarland 1}
\subsection{Kurzbeschreibung}
\begin{itemize}
	\item Name: 
	\item Lage:
	\item Topographie:
	\item ansässige Rassen:
	\item Regierung:
	\item Wirtschaft:
	\item politisches Verhältnis zu unserem Land: neutral, potentiell feindlich
\end{itemize}

\subsection{Geschichte}
Das einstige Königreich (NaLa) wird derzeit von einer Katastrophe in die andere gejagt.
Zuerst kam es zu Aufständen gegen die Monarchie, woraus sich ein handfester Bürgerkrieg entwickelte.
Im Verlaufe dieses Bürgerkrieges kam es zu Enteignungen und viele junge Menschen verloren ihre Eltern.
Diese Situation wird nun schamlos ausgenutzt von einer militanten Bewegung (MilBe), die sich zum Ziel gesetzt hat, aus NaLa einen egalitären Staat zu machen.
Das bedeutet zwar grundlegend, dass alle Einwohner den gleichen Zugang zu den zentralen Ressourcen haben (Nahrungsmittel, Güter, Land usw.) und auch Keiner dauerhaft Macht über Andere ausüben kann.
Der soziale Status des Einzelnen soll vor allem von seinen Fähigkeiten und seinem Willen abhängen.
Es soll politische und soziale Gleichheit herrschen.
Aber damit einher geht auch, dass individueller Besitz und Eigentum nur einen nachrangigen Stellenwert haben. (Im Grunde also das Prinzip des Kommunismus.)\\
\\
Die MilBe rekrutiert viele der elternlosen Jugendlichen, die ohne Bindungen dastehen sowie Personen, die aufgrund der Enteignungen und des Bürgerkrieges keine Verbindungen zu anderen Gemeinschaften mehr haben.
Außerdem verfolgt die MilBe eine stark nationalistische Philosophie, möchte Minderheiten im eigenen Land unterdrücken und ausmerzen und das eigene Volk von Fehlerhaftigkeit befreien.
Zu einer der zu bekämpfenden Fehler gehören alle Magiebegabten, die sich gegen die MilBe stellen, da sie über zu viel Macht verfügen.
In den eigenen Reihen hingegen sieht die MilBe sehr gerne, mächtige und regimetreue Magier.\\
\\
Der letzte Monarch wurde vor kurzem durch einen blutigen, durch die MilBe angeführten Aufstand ins Exil (unser Land oder ein anderes?) vertrieben.
Jetzt beginnt die MilBe die politische Säuberung von NaLa und verfolgt, foltert und ermordet Anhänger, Sympathisanten und Unterstellte der Monarchie und des Monarchen.
Die Hauptstadt von NaLa ist bereits eingenommen und eine „Demokratie“ ist ausgerufen worden, mit dem Anführer der MilBe als Staatsoberhaupt.\\
\\
Nach den langen Kämpfen begrüßt ein Großteil der Bevölkerung nun die Truppen der MilBe jubelnd.
Ein großer Teil der Kämpfer bestand aus Kindersoldaten, die zu diesem Zeitpunkt nichts anderes als ein Leben als Soldat kannten.
Doch diese Stimmungslage kippt sehr schnell als das Staatsoberhaupt mit seinem Terrorregime beginnt.
Die Besonderheit daran: Im Gegensatz zu anderen Diktatoren will dieser sich nicht zu erkennen geben und verbirgt sich hinter den Reihen der MilBe und einer großen Regierungsstruktur.
Damit will er vermeiden, dass Anschläge gegen ihn ausgeführt werden.\\
\\
In dieser Zeit, in der das Terrorregime beginnt, befindet sich NaLa aktuell.\\
\\
Die Hauptstadt wurde komplett „evakuiert“ und die MilBe hat sich dort niedergelassen.
Wer im Verdacht steht, mit Magiern, Minderheiten wie den Zwergen, Monarchieverfechtern oder sogar Ausländern/anderen Rassen zu kollaborieren, wird mitsamt der ganzen Familie (Ehegatten und Kinder) ermordet.
Anfangs werden sie klassisch hingerichtet, doch das Metall ermüdet schnell und die MilBe will Ressourcen sparen.
Darum werden die Menschen (soll es diese oder eine andere Rasse sein?) brutal zu Tode geprügelt.
Kinder werden einfach mit den Köpfen gegen Bäume geschlagen oder von Klippen gestoßen.
Erwachsene werden gehängt, oder je nach belieben anderweitig waffenfrei ermordet. Zuvor jedoch werden alle maßlos gefoltert, um Informationen über weitere Gegner der Regimes zu erhalten.
Gefangenenlager werden errichtet und täglich hunderte Menschen(?) getötet.\\
\\
Alle übrigen Mitglieder der Gesellschaft, die nicht zur MilBe überlaufen, werden in Arbeitslager gezwungen und der bisherige Reichtum, die Infrastruktur und alles, was sich bisher in NaLa entwickelt hatte, soll niedergerissen und neu strukturiert und aufgebaut werden.
Ganz nach den Vorstellungen der MilBe.\\
\\
Aufgrund dieser Entwicklungen ist NaLa gerade vollauf mit seinen eigenen Problemen beschäftigt und wird sich aktuell weder mit unserem Land verbünden noch sich anfeinden – zumindest vorerst!
Allerdings werden mittellose Flüchtlinge in unser Land kommen, die von dort berichten und Schreckliches erlitten haben.

\subsection{Gräueltaten}
Die MilBe-Soldaten, meist jung und vom Land, trieben die Menschen(?) ohne Gnade aus der Hauptstadt.
So auch Kranke, die teilweise noch verbunden waren und deren Binden von Blut durchtränkt, teils von Angehörigen auf ihren Bahren stadtauswärts getragen, andere wiederum versuchten auf allen Vieren zu krabbeln, weil sie keine Verwandten mehr hatten und nicht mehr gehen konnten.\\
\\
Neben den Massenmorden ist eine weitere Methode der MilBe der Tod durch Arbeit.
In den Arbeitslagern müssen die Gefangenen bis zur letzten Kraft arbeiten, wer nicht mehr kann, wird aussortiert oder zum Sterben zurückgelassen.
Viele sterben in den Lagern an Unterernährung, Seuchen, Überarbeitung und Folter.\\
\\
Wer in den Lagern ankam, wurde so lange gefoltert, bis er mindestens 50 bis 60 Personen denunziert hatte.
Diese wurden dann wiederum gefangen genommen usw.\\
\\
Zwangsehen um Verwandtschaftsverhältnisse zu schwächen und den Gehorsam gegenüber der MilBe zu stärken.
Bei der Zeremonie müssen die Eheleute sagen: „Ich danke MilBe, dass sie mir gute Eltern sind und mir erlauben, einen Partner zu haben und sich um mich kümmern, als wäre ich ihr biologisches Kind.“\\
Wenn eine Frau Angst hat oder schlichtweg keinen Sex mit ihrem neuen Ehemann haben will, darf sie gefoltert oder eingesperrt werden.
Der Mann kann sich beschweren, dass er um den Vollzug der Ehe betrogen wurde oder aber der Vorfall wird der MilBe von einem Spitzel gemeldet.
Die Folgen des Verstoßes sind nicht absehbar, weshalb es in den Augen vieler Frauen besser ist, den erzwungenen Geschlechtsverkehr, der vom Staat verlangt wird, einfach geschehen zu lassen.\\
\\
Babys von Verrätern werden nicht behalten.
Sie könnten später Rache nehmen.
Darum werden sie von Untergebenen der MilBe an den Füßen gepackt und gegen Bäume geschleudert.

\section{Nachbarland 2}
\subsection{Kurzbeschreibung}
\begin{itemize}
	\item Name: 
	\item Lage:
	\item Topographie:
	\item ansässige Rassen:
	\item Regierung:
	\item Wirtschaft:
	\item politisches Verhältnis zu unserem Land: freundlich
\end{itemize}

\subsection{Geschichte}

\section{Nachbarland 3}
\subsection{Kurzbeschreibung}
\begin{itemize}
	\item Name: 
	\item Lage:
	\item Topographie:
	\item ansässige Rassen:
	\item Regierung:
	\item Wirtschaft:
	\item politisches Verhältnis zu unserem Land: freundlich
\end{itemize}

\subsection{Geschichte}

\chapter{Land}
\begin{itemize}
	\item Name: Mantodea
	\item Lage: 
	\item Topographie:
	\item ansässige Rassen: \npref{rasse:mensch}, \npref{rasse:zwerge}, \npref{rasse:sylvan}
\end{itemize}
\section{Stammesverbände}
Einst wanderte ein großer Stamm aus den unwirtlichen Gefilden des Nordens weiter nach Süden.
Sie überfielen bestehende Gemeinschaften, raubten sie aus oder machten sie sich Untertan.
Die Nordlinge ließen sich umgeben von grünen Wäldern und fruchtbaren Böden entlang eines großen Flusses nieder.
Im Nordwesten wurde das von ihnen eroberte Gebiet (Mantodea) von einem riesigen Gebirgszug begrenzt und im Südosten durch einen Wald mit riesenhaften Bäumen und einer ungeahnten Ausdehnung (\npref{formation:gigantus}).\\
\\
Ihre Gruppe teilte sich in kleinere Stämme auf.
Einige blieben auf der Ebene am Fluss, andere zog es in die Berge, auch der dort auffindbaren Rohstoffe wegen.
Dieser lockere Verbund aus Stammesverbänden hielt lange Zeit an, auch ohne feste Struktur.
Das einzige, was sie alle miteinander verbunden hat, war die militärische Stärke.

\section{Der Untergang des dunklen Volks}
\begin{itemize}
	\item Waldvolk, Mischung aus Sylvan und Halblingen - animalisch, klein gewachsen, mit dem Wald verbunden und an ihn angepasst
	\item leben in den Schatten tiefer, undurchdringlicher Wälder
	\item Menschen/Nordlinge bauen immer größere Siedlungen, nehmen Wald weg
	\item Waldvolk stiehlt von Vieh und Feldern aufgrund von Nahrungsknappheit und verteidigen für sie wichtige Teile des Waldes
	\item Unruhen verstärken sich, Eskalation
	\item Bürgerkrieg mit Auslöschung des dunklen Volkes
\end{itemize}

\section{Monarchie}
\begin{itemize}
	\item Heerführer übernimmt Landesherrschaft
	\item baut seine Einflüsse aus und sichert seinen Nachkömmlingen die Thronfolge --> es entsteht eine Erb-Monarchie in der viele Intrigen gesponnen, viele interne Kriege und Machtspiele geführt werden und Brudermord auf der Tagesordnung steht, um an den gewünschten Titel zu gelangen
\end{itemize}

\section{Aufstieg der Religion}
\begin{itemize}
	\item Krieg --> bewaffnete Missionierung
	\item Religion kommt von außerhalb
	\item Religion gewinnt den Krieg
\end{itemize}

\section{Herrschaft des Ordens}
\begin{itemize}
	\item Zersplitterung des Ordens in geistige und militaristische Führung
	\item Aufbau von Ausbildungsstätten für magiebegabte junge Menschen
	\item Aufbau einer Heeresstruktur mit Ausbildung junger Rekruten
	
\end{itemize}

\section{Aktuell}
\begin{itemize}
	\item innerhalb des Ordens außerdem Machtkampf der Magie-Häuser
	\item Orden wirbt auch in Nachbarländern an, hauptsächlich aus bildungsferner und mittelloser Schicht.\\
	Eltern sind leichter mit Naturalien und monetären Mitteln auszuzahlen als reiche, gebildete Familien mit geringerer Kinderzahl
	\item Rebellion gegen Gottglauben und den Klerus bildet sich
\end{itemize}

\section{Nahe Zukunft}

\section{Ferne Zukunft}

\chapter{Tal}

\part{Gesellschaft}
\chapter{Religion}
\section{Übersicht}
\begin{itemize}
	\item Der Glaube an 5 gute und 2 böse Götter.
	\item Das Symbol der Religion ergibt sich wie in Abschnitt \ref{sec:goettersymbol} erklärt aus dem Zusammenspiel der 7 Götter und ist in Bild \ref{fig:goettersymbol} dargestellt.
\end{itemize}

\begin{figure}
	\centering
	\includegraphics[width=0.3\textheight]{Abbildungen/Gesellschaft/Goettersymbol}
	\caption[Göttersymbol]{Das Symbol für die Götter}
	\label{fig:goettersymbol}
\end{figure}

\section{Geschichte}
Die hier erzählte Geschichte ist die vom Klerus verbreitete Sage über die Entstehung der Götter des Streits und der Heimtücke. \\\\
Es ist unklar, wie die Götter entstanden sind. Es geschah irgendwie irgendwann. Auf der noch recht unbewohnbaren Erde formten sie lebensfreundliches Gebiet mit ihrer jeweiligen Magie. Sie erschufen das Leben und ließen ihm seinen eigenen Lauf.\\
Nachdem sie jedoch viel Zeit damit verbracht hatten, sich an ihrem Paradies zu ergötzen, wollten sie auch Wesen nach ihrem Abbild schaffen und so lenkten sie das Leben und die Natur und brachten dadurch die Menschen hervor. Die Menschen, intelligent genug um Kultur aufzubringen, verfielen jedoch in viele Kriege. Mitgerissen aufgrund ihrer Ähnlichkeit und weil sie die Menschen so interessant fanden, integrierten sich die Götter in diese Streitigkeiten. Dabei nahmen sie die Positionen ihrer Menschengruppen an und halfen ihnen und wurden so langsam auch in einen Streit untereinander gezogen, wobei sie ihre negativen Seiten ausspielten.\\
Dies äußerte sich in vielen kleinen Dingen \textbf{(hier Geschichtchen einfügen)}, bis die Götter schließlich merkten, dass es so nicht weiter gehen kann. Sie setzten sich zusammen und kamen zum Schluss, dass sie, um wahre Leiter und Lenker des Lebens zu sein, ihre negativen Aspekte los werden müssten. Andernfalls würden ihre dunklen Seiten sie immer wieder übermannen können, was bereits große negative Folgen für alle Lebewesen hatte und auch wieder haben könnte.\\
Darum bereiteten die Götter ein Ritual vor, um perfekt zu werden und sich wortwörtlich von ihren schwachen Seiten rein zu waschen. Dazu begaben sie sich in einen See weitab der Zivilisation. Sie wuschen ihre schlechten Seiten von sich ab und versiegelten diese im See. \textbf{Allerdings gibt es ein kleines Leck, durch welches kontinuierlich ein wenig von der Magie und dem Schlechten austritt.}\\
Danach treten die Götter lange Zeit nicht in Erscheinung. In der Nähe des Sees, den Bach hinunter, lebte ein Paar. All sein Trinkwasser nahm es aus dem Bach und all ihre Ernte wurde mit dem Wasser bewässert. So nahmen sie über Zeit immer mehr von dem Bösen in sich auf. So wie ihre Macht und Magie wuchs, tat es auch das Böse in ihnen und verzehrte ihre guten Geister. Sie begannen aktiv das Böse aus der Umwelt und schließlich dem See zu absorbieren und wurden immer mächtiger und verdorbener. Sie führten Verwüstung und Zwist herbei und die Götter wurden auf sie aufmerksam. Ihre Versuche, sie zu bekämpfen und unter Kontrolle zu bekommen, scheiterten unter großen Verlusten in der Umwelt. Schließlich sahen die Götter ein, dass sie die beiden nicht besiegen konnten und unter all ihrer gebündelten Macht, um alle Lebewesen zu schütze, ersannen sie einen radikalen Plan.\\
Da die Götter zu verantworten hatten, dass diese Monster auf die Welt kamen, sahen sie es als ihre Pflicht, die Welt auch wieder von ihnen zu befreien. Aufgrund der zuvor genannten Umstände war die Maßnahme jedoch von drastischer Natur: sie hoben den Ort, auf dem sie die beiden bekämpften, aus der Erde heraus in den Himmel. Dort werden sie nun für alle Zeit gegen die beiden kämpfen und sie in Schach halten, damit das Leben auf der Erde geschützt ist.\\
Doch die beiden Bösen versuchen ihre Macht zu vergrößern, indem sie ihre Aktuelle auch dazu nutzen, den Lebewesen auf der Erde Böses einzuflüstern und sie zu manipulieren. \textbf{Die Erdenwesen sollen ihnen dienen.}\\
Bevor die guten Götter die Erdmasse mitsamt den Bösen in den Himmel aufgehoben haben, \textbf{erschufen sie gemeinsam einen Führer für die Toten}, der die verstorbenen Seelen zu ihnen bringen soll. Im Laufe der Zeit, in der sie ihre Pflichten als Götter und Führer der Welt etwas außen vor ließen, haben sie es geschafft, die Bösen zurück zu treiben. Nun gibt es nur noch Kampf auf einem kleinen Teil des Himmels-Brockens.\\
Nachdem die Toten verbrannt wurden, steigen ihre Seelen in den Himmel auf. Wenn die Verstorbenen in ihrem Leben gut waren und sich nicht von den Einflüsterungen der bösen Götter haben verführen lassen, werden sie vom Seelenleiter hinüber geführt, auf dass sie mit ihrer Kraft die Götter unterstützen und in ihrem Paradies wohnen können. Die verdorbenen Seelen jedoch werden vom Seelenleiter in die Irre geführt, damit sie nicht den Bösen helfen können, und verlieren sich in der unendlichen Dunkelheit.

\section{Götter}
In dem Land, in dem wir uns befinden, gibt es mehrere Götter. Regional unterscheidet sich die Wahl der bevorzugten Götter.
Diese sind Fiktion und existieren nicht wirklich; es gibt also auch keine Wunder, die von ihnen gewirkt werden. Allerdings interpretieren Menschen ja gerne sehr viel und sehen deshalb ein paar Dinge als Wunder an, die z.B. von den Priestern gemacht werden. Weil diese Priester gute Magier sind, aber das alles natürlich als Geschenk der Götter sehen.\\

\subsection{Das Göttersymbol} \label{sec:goettersymbol}
Beim Wasser-/Reinigungsritual, dem Abtrennungsritual oder einem anderen war die Aufstellung der fünf Götter und ihr Energiefluss wie dargestellt in Abb. \ref{fig:goetteraufstellung}. Denn die stärksten und direktesten Verbindundungen/Folgen ihrer Aspekte sind so. Das ergibt eine liegende acht, also das Symbol für Unendlichkeit --> 8 als heilige Zahl; Götter und ihre Macht sind unendlich.\\
\\
Mit den zwei Bösen Göttern: diese versuchen, alles zu spalten und zu zerstören. Das wird deutlich durch die Linie, welche die 8 versucht zu zerschneiden. Zudem zeigt es aber auch, wie durch die bösen Götter und zum Schutz des Lebens, ein Stück von der Erde abgetrennt werden musste. Des Weiteren mahnt es, wie unsere Schlechten Seiten immer Versuchung und Untergang sind — aber sie sind auch ein Teil von uns und wir müssen lernen, damit umzugehen und ihnen zu widerstehen. Aus diesen Aspekten ergibt sich das endgültige Göttersymbol wie dargestellt in Abb. \ref{fig:goettersymbol}.\\

\begin{figure}
	\centering
	\includegraphics[width=0.7\linewidth]{Abbildungen/Gesellschaft/GoetteraufstellungbeiReinigungsritual}
	\caption{Aufstellung der Götter während des Reinigungsrituals}
	\label{fig:goetteraufstellung}
\end{figure}




\subsection{Gott des Schutzes}
männlich\\
Anführer der Götter\\
positive Aspekte: Mut, Macht, Kraft, Würde, Standhaftigkeit, Entscheidungsvermögen, Führungsqualitäten, Sicherheit, Intuition, Schutz\\
negative Aspekte: rücksichtslos, stolz, unterdrückend, hochnäsig, störrisch, grausam, überheblich, aggressiv, unanpassungsfähig, arrogant, egoistisch
\subsection{Göttin der Harmonie}
weiblich\\
positive Aspekte: Familie, Liebe, Selbstlosigkeit, Gnade, Mitgefühl, Vergebung, Demut, Reinheit, Frieden, Toleranz, Geborgenheit, Harmonie\\
negative Aspekte: aufgeben, rückgratslos, schwach, feige, scheu, abhängig, undiszipliniert
\subsection{Gott des Ausgleichs}
männlich\\
positive Aspekte: Höflichkeit, Aufstieg, Disziplin, Gleichgewicht, Ruhe, Gelassenheit, Geduld, Wachstum, Ausgeglichenheit, Strenge, Heilung, Ausgleich\\
negative Aspekte: Faul, neidisch, gefühllos, unberechenbar, unaufmerksam, charakterlos, jähzornig
\subsection{Gott der Freude}
männlich\\
positive Aspekte: Glück, Hingabe, Enthusiasmus, Fröhlichkeit, Freiheit, Menschlichkeit, Kunst, Kultur, Charme, Kreativität, Freude\\
negative Aspekte: wollüstig, völlernd, frech, eitel, misstrauisch, zynisch, nervös, voreilig, schadenfreudig, chaotisch, instabil, manisch
\subsection{Göttin der Weisheit}
weiblich\\
positive Aspekte: Erleuchtung, Wissen, Verständnis, Konzentration, Wahrheit, Klarheit, (Um-)wandlung, Zielstrebigkeit, Weisheit\\
negative Aspekte: eingebildet, versteift, zwingend, gierig, ängstlich, heimtückisch, habsüchtig, fanatisch, manipulativ, selbstüberschätzend, streitsüchtig
\subsection{Gott der Heimtücke}
männlich\\
positive Aspekte: \\
negative Aspekte: 
\subsection{Göttin des Streits}
weiblich\\
positive Aspekte: \\
negative Aspekte: 

\section{Klerus}
\textbf{Hier soll eine genauere Beschreibung des tatsächlichen Klerus erfolgen, also dem Teil der Struktur, der die Inhalte der Religion übermittelt. Zum Teil, der das Land verwaltet, siehe bitte \npref{ch:regierung}.} 

Da die Fähigkeit zur Nutzung der Magie natürlich auch von den Göttern kommt (nur wenige Arten können dies, ein paar Pflanzen, ein paar Tiere und darunter die menschlichen Arten (und unbekannterweise natürlich auch Mikroorganismen)), ist die Intensität, in der man Magie nutzen kann, natürlich auch ein Zeichen für die Gunst der Götter. Demnach müssen solche Leute stark in der Gunst stehen und daher auch ihr Leben den Göttern widmen - aka in den Orden gehen. Tatsächlich ist das aber auch ein Mittel des Ordens, diese mächtigen Leute zu kontrollieren, da sie die nach ihrer Art prägen und unter ihrer Anweisung haben

\subsection{Aufbau \& Struktur}

\subsection{Riten}
\paragraph{Ideen}
\begin{itemize}
	\item Predigt --> worshipping mit Gesang
\end{itemize}

\subsubsection{Tod}
Wenn jemand stirbt, so muss er verbrannt werden, damit seine Seele vom Körper frei wird und in den Himmel zu den Göttern aufsteigen kann und ihnen im Kampf gegen die bösen Götter beistehen kann. Das sollte optimal dann erfolgen, wenn die zweite Erde am Himmel steht, damit er schnell dort hingelangt. 
Gotteslästerer und böse Leute werden nicht verbrannt, sondern begraben oder ins Wasser geworfen (offenes Meer). So können sie den Bösen Göttern nicht beistehen.
Demnach ist es sehr schlimm, wenn jemand nach seinem Tod nicht angemessen aufbereitet und vor allem verbrannt wird!

\subsection{Gebäude}

\chapter{Regierung} \label{ch:regierung}

\section{Heer}
\begin{itemize}
	\item Der Orden verfügt über ein stehendes Heer. 
		Also eher ein sich ständig bewegendes und im Kampf befindendes Heer, denn es müssen ständig Kontrollen im Land durchgeführt werden, Grenzverteidigung und Niederschlagung von Revolten. 
		Dabei gibt es einige Orte, an denen sich das Heer fest befindet. 
		Tw. haben sich in den Gegenden dafür sogar neue Dörfer oder gar kleine Städte gebildet.
	\item Die Ausbildung erfolgt jedoch in einem sogenannten wandernden Lager. 
		Dieses Lager findet in der Regel immer dort statt, wo die Regierung gerade etwas tatkräftige Arbeiter benötigt. 
		Denn als Teil der körperlichen Ausbildung dürfen die jungen Leute dann kräftig schwitzen: Wälder roden, Seen ausheben, Stadtmauern bauen etc. 
		Jedes Jahr gibt es ein neues Projekt, an dem das Heer sich mit den Azubis beteiligt - zumindest für ein Jahr lang. 
		Danach ist die Umgebung wieder auf sich alleine gestellt.
	\item Die Grundausbildung erfolgt dabei innerhalb von 3 Jahren zum einfachen Soldaten an Lanze, Schwert und kleinem Bogen. 
		In dieser Zeit ist es möglich, sich besonders zu profilieren und ggf. extra Wege angeboten zu bekommen: Kavallerie, Bogenschütze, große Geräte etc. \\ 
		In diesem Fall verlässt man das restliche Heer und kommt zu entsprechenden Ausbildern, die an ganz anderen Orten sein können.
\end{itemize}
\chapter{Kultur}
\section{Rassen}
\subsection{Menschen}
\subsection{Zwerge}

\section{Rollenbilder}
\subsection{Geschlechter}
\subsubsection{Männer} 
Wie auch in unserer Gesellschaft waren die Männer von jeher die Jäger. Daher überzeugen Sie vor allem mit Muskelkraft und Stärke. Aus diesem Grund haben sie sich meistens durchgesetzt, wenn es um Führungsverantwortung geht, denn bisher hat immer der Stärkere gewonnen. Außerdem folgen Sie meistens dem Try and Error Prinzip.
	
\subsubsection{Frauen} 
Frauen hingegen haben als Sammlerinnen eher die Fähigkeit zu beobachten und sich Dinge abzuschauen und zu verbessern. Sie nutzen als Waffe eher ihren Verstand, da sie in der Muskelkraft meist unterlegen sind. Außerdem ist Fingerfertigkeit meist eine Ihrer Stärken. Aufgrund ihrer Eigenschaften als Mutter wird Ihnen auch eher Empathie zugeschrieben. Aus all diesen Gründen sind Frauen in beratenden Stellungen besonders gefragt.
	
\subsubsection{Zusammenspiel}
Das allgemeine Bild in der Gesellschaft ist die Gleichstellung beide Geschlechter. Das eine Geschlecht wäre ohne das jeweils andere nicht lebensfähig. Es werden jeweils die Vorteile aus beiden gezogen, um das Bestmögliche aus der Gesellschaft herauszuholen.

\subsection{Familie}
Auch in der Familie zieht sich das ebenbürtige Bild von Frau und Mann durch. So werden die Aufgaben im Haushalt je nach den Stärken aufgeteilt (bspw. kann ein Mann vermutlich besser Wäsche waschen und eine Frau besser haushalten). Die Aufteilung der Aufgaben ist dabei sogar notwendig, da beide gleichermaßen ins Arbeitsleben eingespannt sein können.
\\
\\
Ebenso verhält es sich mit der Kindererziehung. Je nach den täglichen Tätigkeiten der Eltern können diese mitgenommen und entsprechend erzogen und aufgezogen werden. So werden die beruflichen Fertigkeiten und allgemeinen Qualitäten direkt an die Kinder weitergegeben.

Als Säuglinge werden die Kinder im Idealfall von noch lebenden, aber nicht mehr arbeitstüchtigen Großeltern betreut. Jedoch nur solange, bis die Kinder in der Lage sind, ihre Eltern beim Tagesgeschäft zu begleiten.
\\
\\
\textbf{Exkurs: Homosexualität}\\
Homosexualität ist kein Verbrechen und wird auch nicht als unnatürlich angesehen. Jedoch trifft dies nur solange zu, wie diese Personen trotzdem Nachkommen erzeugen. Handelt es sich um eine kurze Anwandlung in der Jugend gibt es kein Problem. Im Erwachsenenalter braucht es jedoch einen Ehepartner vom anderen Geschlecht, damit Nachkommen gezeugt werden können. EIN Liebhaber vom selben Geschlecht wird geduldet. Lieber soll man so leben, als die Familie zu zerstören. Grundsätzlich ist jedoch nur ein Ehepartner und ein Liebhaber gesellschaftlich akzeptiert.

\subsection{Alter}
Wie im vorhergehenden Punkt beschrieben, können ältere Menschen bei der Kinderbetreuung und -erziehung sehr hilfreich sein. Aus diesem Grund sind sie recht gut angesehen. Außerdem können sie wertvolle Erfahrungen mit Jüngeren teilen. Obwohl sie also keine Arbeit mehr leisten können, tragen sie positiv zur Gesellschaft bei.
\\
\\
Es gibt jedoch einen Punkt, an dem dieses Bild kippt. Werden die Menschen zu alt, dann könnte das einfach Volk denken, dass diese Menschen nie in ihrem Leben richtig arbeiten mussten und lediglich aus diesem Grund noch fit sind. Oder aber sie hatten oder haben einflussreiche Unterstützer, die sie versorgen können. Gerade vom einfachen Volk werden "zu" alte Menschen also eher skeptisch beäugt. Außerdem tragen sie nicht mehr zum Funktionieren der Gesellschaft bei oder brauchen sogar zusätzliche Pflege, was das Bild nicht gerade besser erscheinen lässt.

\section{Geschichten \& Aberglaube}
\subsection{Aberglaube}
Die folgenden Dinge sind nicht wahr, sondern existieren nur als Aberglaube.
\begin{itemize}
	\item \textbf{Hexen:} die Leute haben Angst vor denen, weil diese Magie beherrschen. Zauberer (also Leute, die ihre Kräfte ausgebaut haben) können das ja auch, aber das ist „natürliche Magie“. Hexen besitzen unnatürliche Magie, die nur dazu dient, anderen zu schaden und nicht aus der natürlichen entsteht (Punkt Flüche zB)
\end{itemize}
\chapter{Alltag}

\section{Nahrung}
\href{https://de.wikipedia.org/wiki/Esskultur_im_Mittelalter}{Esskultur im Mittelalter}\\
\href{https://www.grin.com/document/171690}{Obst und Gemüse im Mittelalter}\\
\href{http://www.hortipendium.de/Gem\%C3\%BCsebau_im_Mittelalter}{Gemüsebau im Mittelalter}\\
\href{https://www.leben-im-mittelalter.net/alltag-im-mittelalter/arbeit-und-berufe/bauern.html}{Bauern}\\
\href{https://www.leben-im-mittelalter.net/kultur-im-mittelalter/wirtschaft/landwirtschaft.html}{Landwirtschaft}\\

\chapter{Wirtschaft}
Relevante Links diesbezüglich: \\
\href{https://www.leben-im-mittelalter.net/kultur-im-mittelalter/wirtschaft/handwerk.html}{Handwerk} \\
\href{https://www.leben-im-mittelalter.net/alltag-im-mittelalter/arbeit-und-berufe/handwerker.html}{Handwerker} \\
\href{https://www.leben-im-mittelalter.net/alltag-im-mittelalter/arbeit-und-berufe/handwerker/handwerksberufe.html}{Handwerksberufe} \\
\href{https://www.leben-im-mittelalter.net/kultur-im-mittelalter/wirtschaft/handel.html}{Handel}\\
\href{https://www.leben-im-mittelalter.net/alltag-im-mittelalter/arbeit-und-berufe/bauern.html}{Bauern}\\


\begin{outline}
	\1 Rüstungsschmied
	\1 Waffenschmied
	\1 Allgemeinschmied
	\1 Goldschmied
	\1 Weber
	\1 Winzer
	\1 Bierbrauer
	\1 Jäger
	\1 Abdecker
\end{outline}

\section{Landwirtschaft}


\section{Handwerk}

\section{Handel}

\section{Tal}
\subsection{Berufe}
Eine Zeile heißt ein Beruf, der das alles verbindet. \\
Normale Bevölkerung:
\begin{outline}
	\1 Bauern \& Hirten \& Imker ("`Zeidler"')
	\1 Tavernenwirt \& Brauerei \& Bader \& Müller \& Bäcker
	\1 Schmied \& Wappner
	\1 Jäger
	\1 Köhler
	\1 Holzfäller
	\1 Sammler: Pilze, Kräuter, Beeren
	\1 Fischer \& Netzflicker \& Besenbinder u.ä. (bei schlechtem Wetter)
	\1 Knochenhauer (=Metzger)
	\1 Lederer \& Schuster \& Sattler \& Seiler
	\1 Gerber \& Seifensieder \& Abdecker/Vasner
	\1 Totengräber
	\1 -> die Frauen von Männern mit bestimmten Berufen wie zB Jäger übernehmen die Textilarbeit für das restliche Dorf, Hebamme, ...
\end{outline}

Vom Orden:
\begin{outline}
	\1 Geistlicher mit Familie, dient auch als Schulze/Verweser
	\1 evtl. 2 weitere niedere Geistliche, zB für Grundbildung, etwas Magie?
	\1 eine Einheit Soldaten: Schutz vor Wölfen, Schutz des Gesetzes
\end{outline}



\part{Abenteuer}
\chapter{Hauptcharaktere}
\section{Soldat}
\begin{itemize}
	\item männlich
	\item Alter:\\
	- 16 Jahre während Demo
	\item Aussehen:\\
	- rötliche Haare, bis zu den Augenbrauen\\
	- Sommersprossen\\
	- sonnengebrannte Haut\\
	- strammer Typ
	\item Kleidung:\\
	- praktikabel\\
	- mit etwas Schutz
	\item Charakter:\\
	- bodenständig
	\item Abstammung:\\
	- Vater: hochdekorierter Militär, nie Zuhause. War mal im Dorf stationiert -> dadurch mit Mutti zusammen gekommen\\
	- Mutter: stammt aus dem Dorf, liebevolle Beziehung\\
	- Geschwister: mind. 1 offen
	\item Hintergrund:\\
	- aufgewachsen im Kerndorf\\
	- hohes Interesse an den stationierten Soldaten, verbringt Freizeit auch gerne bei denen\\
	- eifert seinem Vater nach, will aus dem Dorf raus\\
	- eher wohlhabend
\end{itemize}

\section{Magierin}
\begin{itemize}
	\item weiblich
	\item Alter:\\
	- 17 zur Demo
	\item Aussehen:\\
	- wie dargestellt in Abb. \ref{fig:magierin}
	\item Kleidung:\\
	- legt Wert auf Äußeres\\
	- lange Kleider
	\item Charakter:\\
	- "Mädchen"
	\item Abstammung:\\
	- Mutter: die Geistliche im Amt, eingereist\\
	- Vater: deren Mann, Städter, musste mit seiner Frau mit. Während der Char ein Kind ist, ist er noch sehr unglücklich, später hat er sich mit einigen Männern angefreundet und sich mit der Situation zurechtgefunden. Arbeitet bei einem der innerdörflichen Berufe mit\\
	- Geschwister: mind. 3 offen
	\item Hintergrund:\\
	- absolut gläubig erzogen\\
	- eifert ihrer Mutter darin nach, eine hohe Position im Orden zu erlangen und diesem weiter zu helfen
\end{itemize}

\begin{figure}
	\centering
	\includegraphics[width=0.3\textheight]{Abbildungen/Abenteuer/Hauptcharaktere/magierin}
	\caption[Konzeptart Magierin]{Konzeptart Magierin}
	\label{fig:magierin}
\end{figure}


\section{Spionin}
\begin{itemize}
	\item weiblich
	\item Alter:\\
	- 16 zur Demo
	\item Aussehen:\\
	- wie dargestellt in Abb. \ref{fig:spionin}\\
	- Zopf bis Brust
	\item Kleidung:\\
	- praktisch\\
	- Bogen, Dolch (zum Ausnehmen Wild) am Gürtel\\
	- Gugel?
	\item Charakter:\\
	- praktisch veranlagt\\
	- Wildfang
	\item Abstammung:\\
	- Vater: Jäger\\
	- Mutter: \\
	- Geschwister: einige offen
	\item Hintergrund:\\
	- arme und ungebildete Familie\\
	- aufgewachsen am Rand des Dorfes oder ggf. am Waldrand bei den 2-3 Jägerhütten
\end{itemize}

\begin{figure}
	\centering
	\includegraphics[width=0.7\linewidth]{Abbildungen/Abenteuer/Hauptcharaktere/spionin}
	\caption[Konzeptart Spionin]{Konzeptart Spionin}
	\label{fig:spionin}
\end{figure}

\chapter{Hauptmission}
\section{Prolog}
Teil der Demo
\begin{itemize}
	\item Quests zur Festlegung der Magieausprägungen des Spielers
	\item Quests zur Erlernung von Fähigkeiten
	\item Lernen der Questarten
\end{itemize}
\section{Kapitel 1: Jugend}
Teil der Demo
\begin{itemize}
	\item Die 3 Abenteurer lernen sich kennen und erleben erste gemeinsame Abenteuer.
\end{itemize}
\section{Kapitel 2: Ausbildung}
Tutorials zu den Fertigkeiten
\section{Kapitel 3: Kampf für den Orden}
erste Quests
\section{Kapitel 4Forschung zur Magie in Gegenständen}
\begin{itemize}
	\item Spieler stößt auf Ritual/Erz wodurch man Gegenstände mit Magie verbinden kann
	\item Zuspitzung der Situation mit den Rebellen
\end{itemize}
\section{Kapitel 5: Erschaffung eines Gottes}
\begin{itemize}
	\item Quests gegen die Rebellen: Entscheidung für Erfüllung oder Missachtung der Aufträge bleibt offen
	\item Erschaffung des Gottes, was das Leben aller in einem näheren Umfeld aussaugt (schnelle Alterung), alle anderen sind ihrer Magie beraubt allerdings wirkt auch keine Magie mehr auf sie
	\item Spieler wird zum Propheten des Gottes
\end{itemize}


\part{Mechaniken}
\chapter{Regeneration}
Bei allen Mechaniken sollte der Fakt bedacht werden, dass wir ein Rollenspiel mit Fokus auf die Geschichte und Entwicklung wollen. Der Kampf soll keine primäre Rolle spielen wie bspw. in Skyrim.

\section{Leben}
\begin{itemize}
	\item der Spieler erhält eine sehr kleine Auto-Regeneration. Die natürliche Selbstheilung. In ruhigen Momenten (z.B. Schlaf) regeneriert er pro Stunde zB 2 Prozent seines maximalen Lebens. So können kleine Kampfwunden auch von alleine heilen.
	\item weitere Möglichkeiten zur Heilung: Das Aufsuchen von Heilern und Essen.
	\item Heiler können einen nach entsprechender Gebühr und in einer Zeitspanne, die abhängig von ihrer Expertise ist, komplett voll heilen
	\item Durch Essen lässt sich maximal 60 Prozent des Effektivschadens am Gesamtleben heilen. Verliert der Spieler mehr als 60 Prozent Leben, dann ist die Verletzung entsprechend intensiv und signifikant und soundso viele Prozentpunkte müssen von einem Heiler behandelt werden (permanenter Schaden bis geheilt).
\end{itemize}

\section{Mana}
Soll es Mana geben? Oder nutzen wir Cooldowns? Oder Spellslots?\\
\begin{itemize}
	\item \textbf{Cooldown}\\
	Wir haben gedacht, dass CD doof ist, weil man die dann auf dem Bildschirm zeigen müsste und das ein wenig das Feeling nimmt und recht MMO Charakter hat.
	\item \textbf{Mana}\\
	Mana wäre eine akzeptable Möglichkeit
	\item \textbf{Spellslots}\\
	Spellslots sind eine interessante Möglichkeit, insofern man damit recht gut verhindert, dass der Spieler Spells enmasse raushaut. Er muss ein wenig wirtschaften. Und Einflüsse durch Gegenstände oder Umgebung und Schlaf und dergleichen ist einfacher umzusetzen.\\
	Dies ist bisher unser bevorzugter Weg.
\end{itemize}



\section{Hunger}
\begin{itemize}
	\item Man muss täglich wenigstens einen Happen zu sich nehmen, sonst bekommt man Debuffs (auf höheren Schwierigkeitsstufen mehr zB dreimal täglich ausreichend viel, auf niedrigeren zB gar nichts). Je länger, desto heftiger die Debuffs
	\item Man hat einen vergrößerten, aber nicht überdimensionierten Magen. Das bedeutet: Stopft man viel zu viel Essen in sich hinein, bekommt man ab einer gewissen Schwelle nach den 100 Prozent "Magen voll" (zB bei 150 Prozent) Debuffs, weil man Bauchschmerzen hat. Stopft man dann noch mehr in sich rein (zB bis 200 Prozent), dann muss man sich übergeben und verliert alles, was noch im Magen ist, und nimmt zudem etwas Schaden (durch das heftige Kotzen). Zudem verweigert der Spielercharakter für zB eine Woche das Essen des i-Tüpfelchens (was das Brechen dann ausgelöst hat), weil ihm schon bei dem Gedanken schlecht wird. 
	\item Die Magen-Anzeige im Spiel umfasst natürlich nur 0-100 Prozent
	\item Regeneration durch Essen erfolgt sekündlich immer um den gleichen Betrag (nicht Prozent!) - wie lange wird durch das jeweilige Essen festgelegt. So wird eine Gurke zB nur für 1s heilen, eine Buttercremetorte oder ein Fasanenbraten hingegen für 30s. Den Magen füllen sie ebenso unterschiedlich. 
	\item Idee für Erweiterung des Soldaten-Regenerationsskills: ein Punkt kann sein, dass er gelernt hat, sein Essen bedachter zu verspeisen und besser zu verwerten. Das führt zu einer Erhöhung der Heilung pro Sekunde
	\item Idee: jeder/einer/zwei der Start-Charas könnte eine Unverträglichkeit bzw "schmeckt nicht" haben. Das führt dazu, dass man mit diesem Chara dieses Essen nicht nutzen kann bzw. im zweiten Fall dieses Essen nur einen minimalen Bruchteil der Regeneration bietet. Und zB nach einer gewissen Menge, die man das in sich stopft (unabhängig vom 150 Prozent Magen), dann auch Debuffs dazu kommen\\
	man kann das natürlich festschreiben oder wir lassen den Spieler am Anfang des Spiels selbst entscheiden, was es wird, indem das ein Teil einer Kinder-Quest ist
\end{itemize}
\chapter{Leveling und Skills}
\section{Leveling}
\begin{itemize}
	\item Es gibt verschiedene Arten von Skills. Manche können erlernt und etwas gesteigert werden, ohne dass dafür besondere Dinge oder Zeit nötig sind (zB Regeneration). Die anderen wiederum haben ein bestimmtes System aus Selbst-Lernen, Lehrmeistern und Üben
	\item Ich erkläre das System am Beispiel des Schlösserknackens unter der Annahme, dass wir 5 Stufen der Beherrschung eines Skills einführen (zB Neuling, Lehrling, Adept, Meister, Großmeister)
	\item Zu Beginn hat man die Fähigkeit nicht. Es gibt nun zwei Wege, sie zu erwerben:\\
	1. Selbstlehrend. Dabei setzt man sich einfach mal an ein zu knackendes Schloss und egal ob man es schafft oder nicht, man ist nun in der ersten Stufe, Neuling. \\
	2. Lehrmeister. Man kann auch zu einem entsprechenden Lehrmeister gehen (z.B. ein Schlosser) und ihn bitten, einem die Grundlagen beizubringen.
	\item um nun die Fähigkeit zu verbessern sind zwei Abschnitte relevant: Das Üben und der Aufstieg in die nächste Stufe (und damit verbunden neue Fähigkeiten-Möglichkeiten)
	\item erstmal zum Aufstieg. Um die nächste Stufe zu erreichen, ist IMMER ein Lehrmeister (oder auf den niedrigen Stufen meinetwegen auch Lehrbücher) nötig. Dies sorgt auch dafür, dass wir die Relevanz von Geld in unserem Spiel erhöhen.
	In unserem Beispiel würde einem der Schlosser Erläuterungen zu kompizierteren Schlössern geben und sie zeigen.
	\item das Üben ist nötig, bevor man aufsteigen darf. Denn nur, weil ich die Grundschule geschafft habe, heißt das nicht, dass ich bereit für den Master bin. Keiner kann als Kind direkt nach der Erklärung zum Fahrradfahren losradeln und fährt nicht in den nächsten Baum oder kippt um.
	Üben beideutet die Anwendung des Skills in Ausreichender Menge oder Zeit für das Erreichen der kommenden Stufe. Das muss im Allgemeinen während des Spiels erfolgen. Allerdings können da die Lehrmeister - eingeschränkt - helfen. So könnte einem der Schlosser ein paar alte oder falsche oder einfach Übungsschlösser zur Verfügung stellen, damit man an ihnen üben kann. Das kostet natürlich. Macht man dieses Üben unter Aufsicht eines Meisters, dann kommt man schneller voran (man muss zB nicht 20 sondern nur 10 Schlösser knacken). ABER die Menge der Übung, die man bei einem Meister absolvieren kann, variiert je nach aktueller Stufe. So kann man im niedrigsten Bereich alle Übungseinheiten beim Meister absolvieren, doch will man Großmeister werden, dann vllt nur die ersten 20 Prozent...
	\item andere Skills wiederum gibt es nur in einer Steigerungsform: Man kann sie oder auch nicht und vielleicht kann man besser werden, aber nur in einem marginalen Rahmen und dann nur durch Anwendung (zB tanzen)
\end{itemize}

\section{Skills}
\begin{itemize}
	\item Idee für Erweiterung des Soldaten-Regenerationsskills: ein Punkt kann sein, dass er gelernt hat, sein Essen bedachter zu verspeisen und besser zu verwerten. Das führt zu einer Erhöhung der Heilung pro Sekunde
\end{itemize}

\section{Quest-Belohnungen}
\begin{itemize}
	\item viele der optionalen Quests sollen abhängig von ihrer Beschaffenheit dem Spieler wirklich auch eine tolle Belohnung geben: mehr Leben und "mehr" Mana und Resistenzen. Da wir keine Gegenstände haben, in die man solches mittels Verzauberungen legen könnte, ist das eine schöne Möglichkeit, sie dem Spieler trotzdem zur Verfügung zu stellen und ihn damit stärker werden zu lassen
	\item  blöd gesagt: wenn ich eine stinknormale Banditenbande ausnehme, habe ich ja Wunden erfahren, also weiß ich wieder besser mit meinen körpereigenen Regenerationskräften umzugehen und habe ab dann eine leicht erhöhte Regeneration (anstatt 2 Prozent pro h während Rasten oder Schlafen halt 2,5 Prozent oder so). Oder wenn ich in der Quest einem Alchemisten der Uni bei der Mischung und oder dem Ausprobieren eines Trankes helfen soll, steigt meine Resistenz ggü. Giften oder so was. Oder weil ich bei der Banditenbande oben ja gegen die gekämpft habe, habe ich nun mehr Erfahrung im kleinen Ausweichen, weshalb ich es schaffe, häufig weniger Schaden zu nehmen, aka mein Leben erhöht sich etc
	\item  die Rewards normaler Nebenquests sind in der Regel festgelegt von ihrem Wert und ihrer Stufe (bei Waffen und Rüstung). Denn der Bauer wird dir egal wie mächtig du bist, immer nur das gleiche bieten können als Bezahlung. Die Hauptquest, die der Spieler ja auch aktiv erstmal warten lassen soll, um die Nebenquests zu erkunden und zu machen,  wird mitleveln, damit es nicht wie in vielen Spielen ist, dass man sich erstmal an Nebenquests hocharbeitet und die HQ danach nur noch durchrennen und Oneshotten ist und man die Belohnung "das mächtige Schwert des Weisen" direkt zerstört, weil man keinen Platz im Inventar hat, es aber mittlerweile 20 Stufen später weniger wert ist, als der Rest im Rucksack...
\end{itemize}
\chapter{Nutzung der Karte(n)}
\begin{itemize}
	\item Ingame-Map die der Spieler per Tastendruck aufrufen kann
	\item Per Tastendruck (Standard 'M') öffnet sich die Map der Spielewelt. Der Spielercharakter wird als kleines 3D-Figürchen dargestellt
	\item Besuchte Bereiche der Spielwelt werden normal angezeigt
	\item Bereiche die der Spieler noch nicht besucht hat werden durch schwarzen "Rauch" verdeckt
	\item Für Leute, die es nur auf die Story und weniger auf das Erkunden der Welt abgesehen haben, könnte man den Spieler mit seinem Miniatur-Figürchen interagieren lassen - Der Spieler könnte quasi das Figürchen über bereits aufgedeckte Straßen (NUR aufgedeckte Bereiche) ziehen um "Quicktravel" zu simulieren
	\item Eventuell lässt man dann die Ingame-Zeit schneller verstreichen während man diese Mechanic benutzt
	\item Quicktravel sollte nur von vorbestimmten Punkten zu bestimmten Punkten erlaubt sein. Zum Beispiel ein Kutschensystem zwischen den wichtigsten Orten.
\end{itemize}
\include{Dateien/Speichern}

%----------Anhänge------------


\end{document}