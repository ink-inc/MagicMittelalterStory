\chapter{Regierung} \label{ch:regierung}

\section{Regierungsstruktur}
Wie in \ref{ch:klerus} beschrieben, wird die Nation effektiv von den Predigern und Bischöfen der Kirche regiert. 
Ausgebildete Nicht-Magier des Ordens\footnote{sogenannte Gesegnete} unterstehen den Geistlichen und unterstützen sie bei diesen Tätigkeiten. 
Sie haben ein Modikum an Authorität, können aber durch Beratung und ihre Fähigkeiten Einfluss nehmen.\\
Während die Untergebenen eines Bischofs ein hohes Maß an Authorität besitzen mögen (v.A. wenn sie dem Ersten Diener einer Gottheit unterstellt sind), sind sie dennoch nicht über andere Geistliche erhaben und ihnen Respekt schuldig.
Da auf der anderen Seite die Geistlichen ebenfalls dem Bischof unterstellt sind, kann es auch für sie Konsequenzen haben, sich leichtfertig über diese Untergebenen hinwegzusetzen.

\section{Heer}
\begin{outline}
	\1 Der Orden verfügt über ein stehendes Heer. 
		Also eher ein sich ständig bewegendes und im Kampf befindendes Heer, denn es müssen ständig Kontrollen im Land durchgeführt werden, Grenzverteidigung und Niederschlagung von Revolten. 
		Dabei gibt es einige Orte, an denen sich das Heer fest befindet. 
		Tw. haben sich in den Gegenden dafür sogar neue Dörfer oder gar kleine Städte gebildet.
	\1 Die Ausbildung erfolgt jedoch in einem sogenannten wandernden Lager. 
		Dieses Lager findet in der Regel immer dort statt, wo die Regierung gerade etwas tatkräftige Arbeiter benötigt. 
		Denn als Teil der körperlichen Ausbildung dürfen die jungen Leute dann kräftig schwitzen: Wälder roden, Seen ausheben, Stadtmauern bauen etc. 
		Jedes Jahr gibt es ein neues Projekt, an dem das Heer sich mit den Azubis beteiligt - zumindest für ein Jahr lang. 
		Danach ist die Umgebung wieder auf sich alleine gestellt.
	\1 Die Grundausbildung erfolgt dabei innerhalb von 3 Jahren zum einfachen Soldaten an Lanze, Schwert und kleinem Bogen. 
		In dieser Zeit ist es möglich, sich besonders zu profilieren und ggf. extra Wege angeboten zu bekommen: Kavallerie, Bogenschütze, große Geräte etc. \\ 
		In diesem Fall verlässt man das restliche Heer und kommt zu entsprechenden Ausbildern, die an ganz anderen Orten sein können.
\end{outline}