\chapter{Regierung} \label{ch:regierung}

\section{Regierungsstruktur}
Wie in \ref{ch:klerus} beschrieben, wird die Nation effektiv von den Predigern und Bischöfen der Kirche regiert. 
Ausgebildete Nicht-Magier des Ordens\footnote{sogenannte Gesegnete} unterstehen den Geistlichen und unterstützen sie bei diesen Tätigkeiten. 
Sie haben ein Modikum an Authorität, können aber durch Beratung und ihre Fähigkeiten Einfluss nehmen.\\
Während die Untergebenen eines Bischofs ein hohes Maß an Authorität besitzen mögen (v.A. wenn sie dem Ersten Diener einer Gottheit unterstellt sind), sind sie dennoch nicht über andere Geistliche erhaben und ihnen Respekt schuldig.
Da auf der anderen Seite die Geistlichen ebenfalls dem Bischof unterstellt sind, kann es auch für sie Konsequenzen haben, sich leichtfertig über diese Untergebenen hinwegzusetzen.

\section{Armee}
Grobes Konzept:
\begin{outline}
	\1 Der Orden verfügt über eine ständige Armee. 
		Also eher ein sich ständig bewegendes und im Kampf befindendes Heer, denn es müssen ständig Kontrollen im Land durchgeführt werden, Grenzverteidigung und Niederschlagung von Revolten. 
		Dabei gibt es einige Orte, an denen sich die Armee fest befindet. 
		Tw. haben sich in den Gegenden dafür sogar neue Dörfer oder gar kleine Städte gebildet.
	\1 Die einfache Ausbildung erfolgt jedoch in einem sogenannten wandernden Lager. 
		Dieses Lager findet in der Regel immer dort statt, wo die Regierung gerade etwas tatkräftige Arbeiter benötigt. 
		Denn als Teil der körperlichen Ausbildung dürfen die jungen Leute dann kräftig schwitzen: Wälder roden, Seen ausheben, Stadtmauern bauen etc. 
		Jedes Jahr gibt es ein neues Projekt, an dem das Armee sich mit den Azubis beteiligt - zumindest für ein Jahr lang. 
		Danach ist die Umgebung wieder auf sich alleine gestellt.
	\1 Die Grundausbildung erfolgt dabei innerhalb von 3 Jahren zum einfachen Soldaten an Spieß und Kurzschwert. 
		Vielversprechende Kandidaten (und Adelskinder) erhalten nach dem ersten Jahr eine Ausbildung zum Offizier, die berittenen Kampf, Bildung und Taktik beinhaltet.
		Talentierte Rekruten werden zu Spezialisten (zB Bogenschützen, schweres Gerät) ausgebildet.
\end{outline}

\subsection{Aufgaben der Armee}
Der Armee fallen die folgenden Aufgabenbereiche zu:
\begin{outline}
	\item Grenzverteidigung (Sowohl Schutz vor bewaffneten Truppen als auch Schmuggel)
	\item Angriffskrieg 
	\item Befriedung des Landes:\\
	-Zerschlagung von Räuberbanden \\
	-Jagd auf gefährliche Bestien / Rudel\\
	-Niederschlagung von Aufständen gegen den Orden
	\item Ausbildung militärischer Kräfte
	\item günstige Arbeitskräfte (während der Ausbildung)
\end{outline}

\subsubsection{Grundsätzliche Taktiken}
Die Organisation der Armee und der Heerestaktiken sind unter den folgenden Annahmen erstellt:
\begin{outline}
	\item Es gibt die Infrastruktur und das Knowhow, um gute Rüstungen herzustellen (bsp Plattenrüstung). 
		Das macht es lohnend, Soldaten eine gute Ausbildung zukommen zu lassen, da ihre Sterblichkeit nicht zu hoch ist.
	\item Gut trainierte Schlachtrösser sind vorhanden, aber selten (und Rosspanzer existieren). 
		Die Durchschlagkraft und Mobilität berittener Einheiten hat einen wesentlichen Einfluss auf das Schlachtgeschehen, dominiert aber wegen der Seltenheit nicht vollständig.
	\item Es gibt noch keine Armbrüste. Wegen der hohen Durchschlagkraft und Einfachheit des Gebrauchs würden sie einen extremen Einfluss auf den Ausgang des Schlachtverlauf haben.
	\item Starke Kampfmagier sind sehr selten, können aber einen sehr extremen Einfluss auf das Schlachtgeschehen haben. 
		Daher sind sie ein bevorzugtes Ziel für Bogenschützen und erhalten bevorzugt Pferde für erhöhte Mobilität.
	\item Große Feldwaffen (zB Ballisten, Katapulte) existierten und haben einen großen Einfluss auf unbewegliche Strukturen des Gegners
	\item Der Fluss von Informationen (sowohl Befehlskette als auch Aufklärung) ist eher langsam und jeder Vorteil in dieser Hinsicht kann Schlachten entscheiden.
\end{outline}
In einer normalen Feldschlacht entstehen damit die folgenden Beziehungen:
-Kavallerie zerstört leichte Infanterie (=Schild & Schwert)\\
-Kavallerie kann schwere Infanterie (große Schilde/Stangenwaffen) ausmanövrieren oder wird von ihr aufgehalten\\
-Berittene Magier können schwere Infanterie aufreiben\\
-Belagerungswaffen zerstören Gefechtsstellungen (schwere Infanterie/Belagerungswaffen/Bogenschützen/stehende Magier)\\
-Schwere Infanterie beschützt Bogenschützen/stehende Magier\\
-Bogenschützen bedrohen Magier\\
-Magier können Kavallerie aufhalten\\
-Leichte Infanterie/Kavallerie zerstört Bogenschützen im Nahkampf\\
-Bogenschützen zerstören Leichte Infanterie auf mittlere/lange Distanz 

\subsection{Organisation der Armee}
\subsubsection{Hierarchie}
\emph{Das Folgende muss noch mit der tatsächlichen Größe und Wirtschaftskraft der Nation abgeglichen werden.}\\
Um auf verschiedene Bedrohungen schnell reagieren zu können, sind die Stützpunkte der Armee auf verschiedene Teiles des Landes verteilt. 
Dabei ist die Armee in verschieden große Einheiten aufgeteilt. Die kleinste Einheit ist eine Zehnerschaft, die gemeinsam in Formation kämpfen. 
Eine Zehnerschaft ist selten allein zu einer Aufgabe abgestellt. Mit kleineren Aufgaben wird zumeist eine Hundertschaft beauftragt, die neben einfachen 
Fußsoldaten auch die ersten spezialisierten Kräfte beinhaltet.\footnote{Dies können spezialisierte Fernkampf-Einheiten oder Aufklärungseinheiten sowie selten Mediziner, Logistiker und Kampfmagier sein.} 
Ein Heer ist der Verbund mehrer Hundertschaften und kann mehrere Tausend Soldaten enthalten. Es gibt mehrere Heere.\\
Jede Einheit der Armee wird von einem Offizier geleitet, der entsprechende Authoritäten über die ihm unterstellten Einheiten hat.
\subsubsection{Ausbildung}
Die Armee setzt sich aus einem recht kleinen Teil gut ausgebildeter Veteranen und einem recht großen Anteil frischer Rekruten zusammen. 
In 5-Jahres-Zyklen gibt es eine Rotation der Streitkräfte: neue Rekruten werden in allen Teilen des Landes angeworben 
(und zum nächstgelegenen Ausbildungslager) gebracht und ältere Krieger können aus der Armee austreten\footnote{Die meisten einfachen Rekruten 
nehmen diese Gelegenheit nach ihren ersten 5 Jahren war und kehren als Reservisten in ihre Heimat zurück.}. Ausgebildete Soldaten, die später die Armee verlassen, 
finden häufig als Stadtwachen oder persönliche Wachen von Adligen oder Händlern Anstellung.\\
Die Ausbildung findet in zwei Phasen statt: Im ersten Jahr erhalten Rekruten grundlegendes Training - dies ist vor Allem körperliche 
Ertüchtigung und erstes Training mit dem Kampfspieß (schwerer Anti-Kavallerie-Speer). 
Nach einem halben Jahr beginnt das erste Training in Formation. Die Rekruten werden trainiert, sich als Einheit zu formieren, gemeinsam zu bewegen und zu kämpfen. 
Nach dem ersten Jahr Ausbildung stehen die ersten Prüfungen an: Rekruten werden auf ihre Kampffähigkeit, Führungskraft und Loyalität getestet. 
Denjenigen, die in den Tests herausragen, wird eine Offizierslaufbahn angeboten.
\footnote{Alternativ können Rekruten auch durch Empfehlung ihrer Ausbilder dieses Angebot erhalten. So werden teilweise Adlige in die Offizierslaufbahn gehoben.} 
Diejenigen, die sich für das Offizierstum entscheiden, erhalten eine spezielle zweijährige Ausbildung an der Heeresakademie. 
Diese Aubildung beinhaltet grundlegende Bildung (zB Schreiben, Lesen, Rechnen und Etikette), taktisches Wissen sowie Training im berittenen Kampf. 
Die einfachen Rekruten werden weiterhin für Arbeitsprojekte eingesetzt und neben dem Kampfspieß auch im Umgang mit Schild und Kurzschwert trainiert. 
Dafür talentierte Rekruten erhalten stattdessen eine spezialisierte Ausbildung mit Bogen oder Belagerungswaffen.\\
Truppenformationsübungen und Drills finden regelmäßig statt, um das gemeinsame Vorgehen zu festigen. 
Die Offizierskandidaten nehmen auch mindestens einmal jährlich an den Drills teil. \\
Nach drei Jahren Ausbildung gelten die Rekruten als vollwertige Soldaten und werden als gemeinsame Truppe im Land eingesetzt.

\subsection{Status der Armee}
In der Hierarchie des Staates sind die verschiedenen Heere den Geistlichen des Landes unterstellt, auf dem sich ihr Lager befindet. 
Prinzipiell sind Priester den Zehner- und Hundertschaften gegenüber weisungsberechtigt, da die meisten solchen Gruppen sich aber auf Befehle von höherer Stelle breufen können, 
wird diese Authorität selten in größerem Maße eingefordert. 
Heeresgruppen und die sie anführenden Generäle können nur von den Bischöfen der jeweiligen Region angewiesen werden.\\
In den gehobenen Gesellschaftsschichten wird der Dienst in der Armee als ehrenhafte Beschäftigung betrachtet, der sich Kinder außer dem Erstgeborenen widmen können, wenn eine Beschäftigung als Geistlicher nicht möglich ist.
Die unteren Bevölkerungsschichten betrachten die Armee je nach Region deutlich negativer - die Armee ist genauso häufig der Beschützer der Armen vor Banditen und Invasoren wie sie die geballte Faust ist, die auf Wunsch der Kirche Rebellionen niederschlägt. 
Für die Jugend ist eine Einberufung in die Armee allerdings auch eine Gelegenheit, ohne Magie Karriere machen zu können.
\subsubsection{Im Kriegsfall}
Wird der Kriegszustand ausgerufen verschieben sich die Authoritäten der Armee deutlich. 
Sämtliche Heere sind im Krieg ausschließlich dem Ersten Diener Asdir untergeben und müssen den Geboten der Bischöfe nicht mehr gehorchen.
\footnote{Ohne triftigen Grund kann das Verweigern solcher Gebote aber dennoch Konsequenzen haben.}
Propaganda und die Schrecken des Krieges verbessern in solchen Zeiten das Image der Armee maßgeblich.

