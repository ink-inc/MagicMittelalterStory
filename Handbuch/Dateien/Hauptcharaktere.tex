\chapter{Hauptcharaktere}

Altersangaben von anderen Personen beziehen sich immer auf Kapitel 1, wenn nicht ausdrücklich anders angegeben.

\section{Sigran} \label{sec:mc-soldat}
\begin{outline}
	\1 Alter:
		\2 Prolog: 4
		\2 Kapitel 1: 13
	\1 Aussehen:
		\2 dargestellt in Abb. \ref{fig:mc-soldat}
		\2 rötliche Haare, bis zu den Augenbrauen
		\2 Sommersprossen
		\2 sonnengebrannte Haut
		\2 strammer Typ
	\1 Kleidung:
		\2 praktikabel
		\2 mit etwas Schutz
	\1 Charakter:
		\2 bodenständig 
		\2 optimistisch 
		\2 charmant
		\2 ungeduldig
		\2 aufbrausend -> Charentwicklung umfasst ggf. das Unterdrücken dessen
		\2 beharrlich/stur: wenn er sich einmal entschieden hat, dann bleibt er dabei, wenn nichts drastisches passiert
	\1 Abstammung:
		\2 Familie:
			\3 gehören zum Mittelstand (Müller bzw. Geld vom Militär)
			\3 Eltern haben mit 13 rumgepimpert und Mutter wurde versehentlich schwanger -> mussten heiraten
			\3 Vater ist trotzdem seinem Wunsch nach Militär nachgegangen, konnte aber deswegen nicht komplett "`raus aus diesem blöden Dorf"', weil ja noch Familie
			\3 An sich ist der Soldaten-Beruf im Land sehr angesehen. Im Dorf herrschen jedoch gespaltenen Meinungen darüber, ob es besser ist, sich um die Familie oder um das Wohl des Landes zu kümmern. Vor allem die Mütter aus Yanas Generation leiden mit ihr (Grett Hirte, Clara Müller – ihre Schwester, Hedwig Holzfäller)
		\2 Vater Lothar, 52: 
			\3 aus Familie "`Fischer A"'
			\3 war: hochdekorierter Militär, nie Zuhause -> trotzdem großes Vorbild von Sigran
			\3 Kapitel 2: wir erfahren, dass er wohl als Verräter eingesperrt wurde und nach einer Untersuchung dann hingerichtet werden soll -> Motivation für Sigran, tatsächlich zum Militär zu gehen und vor allem dort hohe Positionen zu bekleiden, um Informationen zu erhalten
			\3 Kapitel 4: Sigran erfährt durch seine Recherchen, dass sein Vater tatsächlich das Land/den Staat verriet und zu Recht eingesperrt ist -> moralisches Dilemma ggf auch dadurch, was Lothar gemacht hat
		\2 Mutter Yana, 52:
			\3 stammt aus dem Dorf: Teil der Müllersfamilie
			\3 liebevolle Beziehung zwischen Sohn und Mutter, praktisch alleinerziehend, arbeitet bei Sigran da aber eng mit Tochter zusammen
			\3 wollte für Sigran "`besseres"' Leben als Militär und vor allem auch nicht, dass er sie und das Dorf verlässt, wird aber darüber hinweg kommen, immerhin ist ihre Tochter + Enkel geblieben
			\3 stirbt zwischen Kapitel 1 und 2
		\2 Schwester Marlene, 38: 
			\3 mittlerweile selbst verheiratet mit Peter und hat mehrere Kinder, mit denen Sigran aufwächst, als wären es seine Geschwister
			\3 wirft es dem Vater vor, dass er sie "`verlassen"' hatte und die Mutter alleine mit ihr zurückließ
			\3 Prolog: Marlene hat große Sorge um ihre Kinder und Sigran, da ihre Älteste erst kürzlich verstorben ist durch einen Wolfsangriff beim Spielen im Wald.
	\1 Hintergrund:
		\2 aufgewachsen im Kerndorf
		\2 hohes Interesse an den stationierten Soldaten, verbringt Freizeit auch gerne bei denen
		\2 eifert seinem Vater nach, will aus dem Dorf raus
	\1 Freunde 
		\2 Prolog
			\3 Kerngruppe: Neffe Waldemar (Müller 4/5), Nichte Geertje (Müller 3/4) und Sigran (Soldat 4/5) mit Anhängsel Helfricus (Hirte 5/6). Helfricus hängt Sigran am Rockzipfel, weil sein Papa Soldat ist und Helfricus das unbedingt auch werden möchte. Sigran ist von seinem Schatten aber wenig begeistert.
			\3 erweitert: Faeladore (Priester 5/6), Quirinor (Priester 3/4), Ernest (Jäger 6/7) und Kastija (Jäger 4/5), Baso (Jäger 8/9) immer, wenn er kann
			\3 Die Soldaten-Kinder spielen zwar gerne mit den anderen, müssen aber, wenn es um den Wald geht, immer daheimbleiben. Manchmal schleichen sie sich aber mit hinaus. Marlene sieht es nicht gerne, wenn Sigran mit Kastija und ihren Geschwistern spielt, weil sie häufig in den Wald gehen und ihr zu draufgängerisch sind.
		\2 Kapitel 1
			\3 mittlerweile gut befreundet mit Helfricus (Sigran und er wollen gemeinsam zum Militär)
			\3 Gruppierungen mittlerweile aufgelöst und es gibt vor allem einzelne engere Beziehungen und ansonsten sind alle Jugendlichen miteinander bekannt und machen was und man mag halt individuell manche mehr und manche weniger		
\end{outline}

Familie:
\begin{outline}
	\1 Vater
	\1 Mutter
	\2 1 Schwester
	\3 10 Nichten \& Neffen (nicht alle haben lange gelebt)
	\3 Prolog: N2, N3, N4, (N5)
	\3 Kap1: N3, N4, N5, N7, N8, (N9)
\end{outline}


\begin{figure}[tbh]
	\centering
	\includegraphics[width=0.75\textheight]{Abbildungen/Abenteuer/Hauptcharaktere/soldat.png}
	\caption[Konzeptart Soldat]{Konzeptart Soldat als 16-Jähriger während der Ausbildung in Kapitel 2.}
	\label{fig:mc-soldat}
\end{figure}




\section{Faeladore} \label{sec:mc-diplomatin}
\begin{outline}
	\1 Alter:
		\2 Prolog: 5
		\2 Kapitel 1: 14
	\1 Aussehen:
		\2 wie dargestellt in Abb. \ref{fig:mc-diplomatin} 
		\2 gut gepflegt 
		\2 geschminkt, sobald sie erwachsen ist
		\2 eigentlich schöne lange Haare, aber nach Streit mit Mutter in Kapitel 1 schneidet sie die kinnlang ab -> lässt sie später wieder lang wachsen (Mädchen), aber der Spieler kann dann entscheiden, sie ggf. wieder schulterlang zu halten. 
	\1 Kleidung:
		\2 legt Wert auf Äußeres
		\2 lange Kleider
	\1 Charakter:
		\2 "Mädchen"
		\2 ihr ist die Ehre der Familie sehr wichtig 
		\2 legt viel Wert darauf, was andere von ihr halten 
		\2 sehr offen gegenüber den schlechter beglückten
		\2 hilfsbereit
		\2 selbstbewusst \& -bestimmt
		\2 perfektionistisch
		\2 rechthaberisch
		\2 nette Person
		\2 gebildet \& handelt bewusst wenn es um geplante Dinge geht, aber reagiert schnell emotional in der Hitze des Gefechts
		\2 ein wenig tollpatschig (im Sinne von unaufmerksam/unbeachtet)
		\2 romantisch: sucht eher langfristig die Liebe, als das kurze Vergnügen
	\1 Abstammung:
		\2 Familie
			\3 eingereist aus anderen Stadt wegen dem Job der Mutter
		\2 Mutter Eleonora, 42:
			\3 die Geistliche im Amt: sehr gebildet und fähig, ihren Job zu erledigen, aber gerade in Bezug auf die eigene Familie nicht sehr feinfühlig. Kann sich jedoch ansonsten gut in "`ihre Schäfchen"' versetzen und diesen helfen
			\3 (sehr) streng aber gerecht trifft's sehr gut
			\3 hat große und schon ausformulierte Erwartungen an ihre Kinder und deren Zukunft. Erwartet zB von Faeladore, dass sie in ihre Fußstapfen tritt und auch Priesterin wird
			\3 großes Vorbild von Faeladore in Bezug auf was sie alles erreicht hat und wie sie mit Menschen umgehen kann, aber überhaupt nicht in Bezug darauf, wie streng und unbeugsam sie mit ihren eigenen Kindern ist
		\2 Vater Ulrych, 39: 
			\3 deren Mann, Städter, musste mit seiner Frau mit. Während Faeladore ein Kind ist, ist er noch sehr unglücklich mit der Situation, später hat er sich mit einigen Männern angefreundet und sich mit der Situation zurechtgefunden. 
			\3 Arbeitet bei den Kräuterleuten mit, da er eine Ausbildung in der Heilkunst mit Pflanzen hatte -> ist dadurch auch das medizinischste Personal, dass das Dorf hat
			\3 ansonsten ist er auch für die Ausbildung im Schreiben für alle Kinder, die es brauchen, zuständig
			\3 der entspannte Teil der Eltern: dort gehen die Kinder hin, wenn sie ein Ja hören wollen
		\2 Geschwister:
			\3 Quirinor, 12: wird sich später in L4Iris (Holz A) verlieben und deshalb nicht dem Weg folgen, den seine Mutter für ihn will; BFF ist später SZ1Albretch (Kräuter) 
			\3 Rihtelin: stirbt mit 3 Jahren kurz nach dem Prolog
			\3 Amalindis, 7: 
			\3 Askona, -1: 
	\1 Hintergrund:
		\2 absolut gläubig erzogen
		\2 eifert ihrer Mutter nach, allerdings nicht als Priesterin
		\2 Pubertät: Streit mit Mutter, weil sie kein Priester sein will (ihre Art der Rebellion)
	\1 Freunde
		\2 Prolog:
			\3 Kern: Faeladore (Priester 5/6), Quirinor (Priester 3/4), Ernest (Jäger 6/7) und Kastija (Jäger 4/5), Baso (Jäger 8/9) immer, wenn er kann
			\3 erweitert: Waldemar (Müller 4/5), Geertje (Müller 3/4) und Sigran (Soldat 4/5) immer nur, wenn sie nicht im Wald spielen, mit Anhängsel Helfricus (Hirte 5/6)
		\2 Kapitel 1
			\3 BFF Geertje (Müller)
\end{outline}

Familie:
\begin{outline}
	\1 Mutter Priesterin
	\1 Vater Gelehrter
		\2 4 Geschwister (alle jünger)
		\2 Prolog: B2, B3
		\2 Kap1: B2, B4
\end{outline}

\begin{figure}[tbh]
	\centering
	\includegraphics[width=0.325\textheight]{Abbildungen/Abenteuer/Hauptcharaktere/magierin}
	\caption[Konzeptart Diplomatin]{Konzeptart Diplomatin als 17-Jährige während der Ausbildung in Kapitel 2.}
	\label{fig:mc-diplomatin}
\end{figure}





\section{Kastija} \label{sec:mc-spionin}
\begin{outline}
	\1 Alter:
		\2 Prolog: 5
		\2 Kapitel 1: 14
	\1 Aussehen:
		\2 wie dargestellt in Abb. \ref{fig:mc-spionin}
		\2 Zopf bis Brust
	\1 Kleidung:
		\2 praktisch
		\2 Hosen 
		\2 Bogen, Dolch (zum Ausnehmen Wild) am Gürtel
		\2 Gugel?
	\1 Charakter:
		\2 praktisch veranlagt
		\2 Wildfang \& lebhafte Persönlichkeit, aber als Erwachsene unterdrückt sie das
		\2 zynisch, schwarzer Humor
		\2 distanziert/nicht aufgeschlossen gegenüber neuen Personen: es ist schwer, ihr Vertrauen zu gewinnen
		\2 sucht eigentlich nach Anerkennung/Aufmerksamkeit, aber will das nicht raus lassen: versucht was zu schaffen/Eindruck zu erwecken, aber fordert es dann nicht ein, sondern erwartet, dass das Lob von selbst kommt
		\2 schon früh sehr selbstständig, weil sie es sein musste (arme Großfamilie)
		\2 stoisch
		\2 aufmerksam und umsichtig
		\2 schlau und recht rational, behält kühlen Kopf
		\2 keine Jungfrau (ab Kap3): manchmal für den Job nötig, ansonsten hat sie auch gerne mal Spaß (ONS) beim Herumreisen. Ggf. an einem Mann in der Organisation interessiert, drückt sie aber wenn dann nur über Scherze/Sticheleien/Neckereien aus
	\1 Abstammung:
		\2 Familie
			\3 Prolog: Sie leben in einfachen Verhältnissen außerhalb des Dorfes.  \\
			Die Eltern haben ein gutes Verhältnis zueinander. Faeladore, Quirinor, Ernest und Kastija spielen gerne zusammen, Baso muss den Eltern helfen, kommt aber gerne dazu, wenn er Zeit hat. Lorinda ist mit eigener Familie beschäftigt und übernimmt vermehrt die Aufgaben einer Jägerin. Durch ihre erste Schwangerschaft muss ihr Mann aushelfen, ist für die Jagd aber wenig geeignet und kann besser beim Fallenstellen behilflich sein. Rupina eifert ihrer großen Schwester nach und wäre am liebsten auch sofort verheiratet. Sie liebäugelt mit den heiratsfähigen Jungen im Dorf. Besonders Adam (Hirte C) hat es ihr angetan.
		\2 Vater Otto, 50: 
			\3 Jäger
			\3 Otto hat eine schwache Konstitution und kränkelt oft. Er ging einmal bei schwerem Unwetter zur Jagd, hat sich dabei schwer erkältet und davon sein Lebtag nicht mehr so recht erholt. Darum geht Iris jetzt öfter jagen und überlässt vermehrt Otto den Haushalt. 
			\3 Die Kinder haben eine starke Bindung zu ihm und sorgen sich um seine Gesundheit. Im Allgemeinen ist er ein netter und fürsorglicher Vater
			\3 Vermisst das Jagen und ist deshalb ab und an schlecht gelaunt /frustriert. Das versucht er zwar, nicht an den Kindern auszulassen, ist dann aber schnell gereizt und reagiert nicht so lieb wie sonst. 
		\2 Mutter Iris, 45: 
			\3 aus Förster-Familie (A)
			\3 stirbt in Kapitel 1
		\2 Geschwister
			\3 Lorinda, 24: verheiratet mit Anno (26, Schmied). Gerade mit dem 4. Kind schwanger, von den ersten dreien hat nur Conntz überlebt
			\3 Rupina, 21, heiratet Adam (23, Hirte C)
			\3 Wido: stirbt mit 3 Jahren noch lange vor Kastija
			\3 Baso: stirbt zwischen Prolog und Kap1 mit 8 Jahren
			\3 Ernest, 15: 
			\3 Gebhart: stirbt mit etwa 1/2 Jahr an Kinderkrankheit vor Prolog (evtl. kurzes Gespräch zw. Mutter Iris und Lorinda überhörbar)
			\3 Jechli, 7:
			\3 Ratolf, 5: stirbt mit 10
			\3 Lufert, 3:
	\1 Hintergrund:
		\2 arme und ungebildete Familie
		\2 aufgewachsen am Rand des Dorfes oder ggf. am Waldrand bei den 2-3 Jägerhütten
		\2 Der Tod ist in verschiedener Hinsicht ein ständiger Begleiter. 
\end{outline}

Familie:
\begin{outline}
	\1 Mutter Jägerin
	\1 Vater Jäger
		\2 10 Geschwister (5 älter)
		\2 Prolog: K1, K2, K4, K5, K7 (stirbt)
		\2 Kap1: K1, K2, K5, K8, K9, K10
\end{outline}

\begin{figure}[tbh]
	\centering
	\includegraphics[width=0.9\textwidth]{Abbildungen/Abenteuer/Hauptcharaktere/spionin}
	\caption[Konzeptart Spionin]{Konzeptart Spionin als 16-Jährige während der Ausbildung in Kapitel 2.}
	\label{fig:mc-spionin}
\end{figure}
