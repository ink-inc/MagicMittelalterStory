\chapter{Wirtschaft}
Relevante Links diesbezüglich: \\
\href{https://www.leben-im-mittelalter.net/kultur-im-mittelalter/wirtschaft/handwerk.html}{Handwerk} \\
\href{https://www.leben-im-mittelalter.net/alltag-im-mittelalter/arbeit-und-berufe/handwerker.html}{Handwerker} \\
\href{https://www.leben-im-mittelalter.net/alltag-im-mittelalter/arbeit-und-berufe/handwerker/handwerksberufe.html}{Handwerksberufe} \\
\href{https://www.leben-im-mittelalter.net/kultur-im-mittelalter/wirtschaft/handel.html}{Handel}\\
\href{https://www.leben-im-mittelalter.net/alltag-im-mittelalter/arbeit-und-berufe/bauern.html}{Bauern}\\


\begin{outline}
	\1 Rüstungsschmied
	\1 Waffenschmied
	\1 Allgemeinschmied
	\1 Goldschmied
	\1 Weber
	\1 Winzer
	\1 Bierbrauer
	\1 Jäger
	\1 Abdecker
\end{outline}

Die Wirtschaft ist ein weiten Teilen am frühen Hochmittelalter angelehnt. 
Landwirtschaft beginnt einen Überschuss an Nahrung zu produzieren, der größere Städte mit spezialisierten Arbeitern ernähren kann. 
Erste Verbände von Handwerkern (Innungen bzw ``Gilden'') sind gerade in den Städten am Entstehen (was eine größere Mitsprache und Selbstbestimmung \& -organisation erlaubt).\\


\section{Landwirtschaft}
Auf dem Land bestellen die einzelnen Familien ihre jeweils eigenen Äcker. 
Unfreie stehen dabei teilweise unter Aufsicht ihres Verwalters und produzieren nach Vorgaben das, was benötigt wird. 
Zudem werden Unfreie bevorzugt zu größeren Arbeitprojekten des Ordens herangezogen (vA Instandhaltung bestehender Infrastruktur), wenn die Arbeitslager der Armee nicht in der Nähe sind.

\section{Handwerk}
AUf Dörfern (vA Kleineren) existiert keine Ausspezialisierung der Handwerksberufe. 
Meist übernehmen Familien Tätigkeiten für die Dorfgemeinschaft (und bilden dadurch ein gewisses Maß an Know-How), haben aber auch Erfahrungen in den meisten anderen Berufen.\\
In den Städten existieren Handwerker, die sich auf eine Tätigkeit spezialisiert haben und ihre Techniken an ausgewählte Schüler (nicht zwingend verwandt) weitergeben. 
Immer größere Zusammenarbeit, der Austausch von Wissen und Techniken hat dazu geführt, dass Handwerker anfangen, Organisationen zu bilden und die Ausbildung zu regulieren. 
Diese Gilden sind auf eine Stadt beschränkte Organisationen und noch nicht in allen Städten vorhanden.
Im Rahmen dieser Entwicklung haben die Gilden das Recht vom Orden zugesprochen bekommen, den überregionalen Jugendaustausch selbst zu organisieren (Wanderjahre von Gesellen).\\ 
Gilden sind im Begriff, magiebasierte Techniken in größerem Stile zu erforschen und werben dafür auch gezielt Magiebegabte an, die sich gegen ein Leben als Geistliche entschieden haben.

\section{Handel}
Während in anderen Nationen überregionale Händlergruppen (Gilden) bestehen, die großen Einfluss und eigene Privilegien besitzen, ist das in Mantodea weitgehend nicht der Fall. 
Der Orden reguliert und steuert Großhandel in Mantodea, indem nur authorisierte Händler Mantodea betreten dürfen\footnote{Das wird mit Zollstationen und Kontrollen von Händlern realisiert. Es gibt trotzdem Schwarzmärkte und Schmuggler.} 
und diesen Händlern Vorgaben gegeben werden, welche Gebiete mit welchen Gütern zu beliefern sind (um eine faire, wenn auch nicht gute Versorgung der Bevölkerung zu gewährleisten). 
Diese Großhändler erreichen teilweise nur die größeren Städte von Mantodea und die weitere Zirkulation der Güter wird durch Beauftragte des Ordens übernommen.\\
\emph{Geld}: Mit dem Entstehen großer Handelsgilden entstand auch ein größerer Fokus auf Geld als Handelsmittel, und zur Zeit des Spiels ist die Währung einer handelsstarken Seenation im internationalen Handel weitverbreitet. 
In größeren Städten Mantodeas ist die Währung Mantodeas ebenfalls weit verbreitet und akzeptiert. 
Um den Handel von außen zu verhindern, ist der Besitz fremder Währungen verboten und kann auf den Einfluss fremder Nationen hindeuten. 
In Dörfern, vA wirtschaftsschwachen, ist Tauschhandel von Produkten und Dienstleistungen deutlich gängiger.\\~\\
\emph{Schuster und Kesselflicker}: Manche reisende Händler, vor allem für kleinere Dörfern, verdingen sich als eine Art ``Arbeiter für alle Zwecke'': 
Sie schustern, flicken und erfüllen andere Arbeiten, für die es unter Umständen keinen Spezialisten im Dorf gibt. 
Zudem handeln sie mit diesem und jenem Krimskrams, den sie während ihrer Reisen aufgelesen haben. 
Allerdings führt ihr unsteter Lebensstil (kein fester Wohnort, keine ``respektable'' Beschäftigung) dazu, dass sie häufig wenig Respekt erfahren.\\
Idee: der PC kann während des Spiels mehrfach demselben Kesselflicker begegnen. Wenn er ihm mit Respekt begegnet und vllt manche Aufträge für ihn erledigt, kann er von dem 
Kesselflicker Informationen oder einzigartige Gegenstände erhalten.


\section{Tal}
\subsection{Berufe}
Eine Zeile heißt ein Beruf, der das alles verbindet. \\
Normale Bevölkerung:
\begin{outline}
	\1 Bauern \& Hirten \& Imker ("`Zeidler"')
	\1 Tavernenwirt \& Brauerei \& Bader \& Müller \& Bäcker
	\1 Schmied \& Wappner
	\1 Jäger
	\1 Köhler
	\1 Holzfäller
	\1 Sammler: Pilze, Kräuter, Beeren
	\1 Fischer \& Netzflicker \& Besenbinder u.ä. (bei schlechtem Wetter)
	\1 Knochenhauer (=Metzger)
	\1 Lederer \& Schuster \& Sattler \& Seiler
	\1 Gerber \& Seifensieder \& Abdecker/Vasner
	\1 Totengräber
	\1 -> die Frauen von Männern mit bestimmten Berufen wie zB Jäger übernehmen die Textilarbeit für das restliche Dorf, Hebamme, ...
\end{outline}

Vom Orden:
\begin{outline}
	\1 Geistlicher mit Familie, dient auch als Schulze/Verweser
	\1 evtl. 2 weitere niedere Geistliche, zB für Grundbildung, etwas Magie?
	\1 eine Einheit Soldaten: Schutz vor Wölfen, Schutz des Gesetzes
\end{outline}
